\chapter{Analysing results}
\label{chap:Results}
\section{The results XML file}
\index{The results XML file}

After completion of the experiment a file containing results is
always given. The default results file is in the XML format and it
contains all the information about the course of the completed
experiment.

\apex automatically assigns a default name to the results file,
namely it appends ``-apr'' to the name of the experiment file
(e.g. closedsetword-apr). It will never overwrite an existing
results file, but will append a number to results (e.g.
closedsetword-apr-1) in the case of an existing results file.

The results are stored in the element
\begin{lstlisting}
<apex:results>

</apex:results>
\end{lstlisting}

The XML file contains \element{general} information, such as

\begin{itemize}
\item the testing date

\item the testing duration

\item the name of the XSLT script file (see~\ref{sec:Using XSLT
transforms}).

\item information on the procedure (eg: the adaptive parameter)

\end{itemize}

In addition, for each completed \element{trial} presented to the
subject it includes

\begin{itemize}
\item details of the procedure (stimulus that was presented and
the values of the variable parameters in the specific trial).

\item the response that was chosen by the subject

\item the outcome of the corrector

\item possible errors of the output system/device (eg. underruns)

\item the response time (time between the moment the buttons are
 enabled and the moment an answer is given)


\item in case a random generator was used: the value of the random
generator in the specific trial
\end{itemize}

Remark: with an adaptive procedure the number of
\element{reversals} is given

If \element{saveprocessedresults} is \xml{true} in the experiment
file the processed data will be appended to the XML file under
\element{processed}

The results file contains a lot of information. A summary of the
relevant results can be obtained through an XSLT transform (see
next paragraph).

\section{Using XSLT transforms}

An XSLT script transforms the results XML file to a summary of the
results. For each \apex procedure a script is provided, which
processes the results in an appropriate way according to the
procedure (stimulus/response pairs, score, mean threshold of n
reversals,...). You can also create your own scripts, provided you
have some XSLT knowledge.

If XSLT is specified in the experiment file \apex transforms the
data. It is then possible to 1) show the results on the screen
(\element{showresults}=\xml{true}) and/or 2) to save the processed
results to a file (\element{saveprocessedresults}=\xml{true}).
Then it is not necessary to execute the XSLT transform described
in the next paragraph.

\label{sec:Using XSLT transforms}

\oxygen{ To execute an XSLT transform, a transformation scenario
has to be configured once. This is done by clicking on the
``Configure transformation scenario'' button in Oxygen. If you
then click on ``new'', you can search for the appropriate XSLT
script in \filename{scripts/xslt} (for the moment the names of the
scripts correspond to the apex2 procedure names) and give it a
name in the upper field of the ``edit scenario'' box. If you then
click on ``transform now'', the processed results are shown
immediately.}

The processed results can be saved in the results XML file and/or
shown immediately after a test (in the last case the program asks
whether you want to see them or not).

To give stimulus-response pairs, the XSLT transformation not only
needs the results XML file, but also the experiment file. This
experiment file is defined at the top of the results XML file.
Take care that the correct path is given there.
\newline
\newline Example result commands in experiment xml file:
\begin{verbatim}
<general>
    <showresults>true</showresults>
    <saveprocessedresults>true</saveprocessedresults>
    </general>

 <results>
   <xsltscript>idn.xsl</xsltscript>
 </results>
 \end{verbatim}
