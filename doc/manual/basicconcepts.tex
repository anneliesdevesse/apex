\chapter{Basic concepts}
\label{chap:Basic concepts}
\section{Introduction}

An \apex experiment is defined in an experiment file. Experiment
files are defined in the XML format (see section~\ref{sec:xml}),
and it is advised, but not compulsory, to use OxygenXML to edit
the XML format (see section~\ref{sec:XML editor}).

While experiment files can have any filename and any extension, it
is advised to give your experiment file the extension
\filename{.apx}, such that is immediately associated with \apex.
Also, there is a clear distinction between \emph{experiment} files
and \emph{result} files, which will receive the extension
\filename{.apr}.


\oxygen{OxygenXML may not recognize the experiment file with
extension \filename{.xml} immediately. If so, select ``All files''
under ``Files of type'', and select the desired experimental
file.}

In addition, the results file (in XML format) can be transformed
to a summary of results using an XSLT script. See
section{~\ref{sec:Using XSLT transforms}.

\index{Basic concepts}


\section{Terminology}

\apex is based on a few basic concepts that are common to all
psychophysical experiments. Here we define the concepts device,
controller, datablock, stimulus, filter, screen, response, trial,
procedure, experiment, result, ID, and parameter.

\begin{description}
\index{device}\index{controller}\index{datablocks}\index{stimulus}\index{filter}\index{screen}\index{response}
\index{trial}\index{procedure}
\index{experiment}\index{result}\index{ID} \index{parameters}

\item[device] is a system connected to the computer that can be
controlled by \apex. Devices can send signals to a transducer.
Examples are sound cards and interfaces to cochlear implants
(CIs).  Devices can have parameters that are controlled by \apex.
\item[controller] is a system connected to the computer that
-unlike a filter- does not accept/transfer signals (eg gain), but
has parameters that are controlled by \apex. An example is a
programmable attenuator that controls the gain of an amplifier.
\item[datablock] is an abstraction of a basic block of data that
can be processed by \apex and can be sent to a matching device. In
the case of a sound card, the datablock would be an audio signal
in the form of a series of samples that is commonly stored on disk
as a "wave file". \item[stimulus] is a unit of stimulation that
can be presented to the subject, to which the subject has to
respond. In the simplest case it consists of a single
\concept{datablock} that can be sent to a single \concept{device}.
More generally it can consist of any combination of
\concept{datablock}s that can be sent to any number of
\concept{device}s, simultaneously or sequentially. \item[filter]
is a data processor that runs inside \apex and that accepts a
block of data, e.g., a certain number of samples from an audio
file, and returns a processed version of the block of data. An
example is an amplifier that multiplies each sample of the given
block of data by a certain value. \item[screen] is a GUI
(Graphical User Interface) that is used by the subject to respond
to the stimulus that was presented. \item[response] is the
response of the test subject. It could, for example, be the button
that was clicked or the text that was entered via the screen.
\item[trial] is a combination of a screen that is shown to the
subject, a stimulus that is presented via devices and a response
of the subject. \item[procedure] The procedure controls the flow
of an experiment. It decides on the next screen to be shown and
the next stimulus to be presented. Generally, a procedure will
make use of a list of predefined trials. \item[experiment] file: a
combination of procedures and the definitions of all objects that
are necessary to conduct those procedures. \item[result] is
associated with an experiment and contains information on every
trial that occurred. \item[ID] is a name given to an object
defined in an experiment. It is unique for an experiment. If, for
example, a device is defined, it is given an ID by which it can be
referred to elsewhere in the experiment. \item[parameter] is a
property of an object (e.g. a device or filter) that is given an
ID. A filter that amplifies a signal could for example have a
parameter with ID \emph{gain} that is the gain of the amplifier in
dB. The value of a parameter can be either a number or text.
\item[standard] In a multiple alternatives forced choice procedure
the reference signal is defined as the standard. Internally, \apex
does not differentiate between a stimulus and a standard. See
example 2: the standard is defined under \element{stimuli} and
should be referred to by its ID.

\end{description}

\section{XML}

\label{sec:xml}

\subsection{Basic XML: elements, tags, attributes}
\label{sec:BasicXML:elements,tags,attributes}

\index{XML}

Advantages of the XML format are that it is human readable, i.e.,
it can be viewed and interpreted using any text editor, and that
it can easily be parsed by existing and freely available
parsers\footnote{\apex\ uses the Xerces-c parser for parsing XML
files. \url{http://xerces.apache.org/xerces-c/}}. Moreover, many
tools exist for editing, transforming or otherwise processing XML
files. In addition, a good XML editor can provide suggestions and
documentation while constructing an experiment file. An XML file
consists of a series of \emph{elements}. Elements are started by
their name surrounded by \xml{<} and \xml{>}, the begin tag, and
ended by their name surrounded by \xml{</} and \xml{>}, the end
tag.
\begin{lstlisting}
    <b>1</b>
\end{lstlisting}


Every element can have content. There are two types of content:
\emph{simple} content, for example a string or a number, and
\emph{complex} content: other elements. An element can also have
attributes: extra properties of the element that can be set.
 In the following example, element \element{a} is started
on line 1 and ended on line~\ref{xml:end}. Element \element{a}
contains \emph{complex} content: the elements \element{b} and
\element{c}. Element \element{b} contains \emph{simple} content:
the numerical value 1. Element \element{c} again contains
\emph{complex} content: the elements \element{c1} and
\element{c2}. Element \element{c1} has an attribute named attrib1
with value 15. Element \element{c2} on line~\ref{xml:empty} shows
the special syntax for specifying an empty element. This is
equivalent to \xml{<c2></c2>}. The reference manual lists all
possible elements and attributes.

\lstset{escapeinside={(*@}{@*)}}
\begin{lstlisting}
<a>
    <b>1</b>
    <c>
        <c1 attrib1="15"> </c1>
        <c2/>   (*@\label{xml:empty}@*)
    </c>
</a> (*@\label{xml:end}@*)
\end{lstlisting}

The begin tag and end tag are case sensitive. A comment is
inserted as follows:
\begin{lstlisting}
 <!--
 This is a comment.
  -->
\end{lstlisting}

\oxygen{ If you start typing a name of an element in OxygenXML,
several options will be offered. Choose one, press ``Enter'' and
add the content.}

\index{element}

\index{begintag, endtag}

\index{child element}

A document that complies with the XML specifications is
``well-formed''. A document is checked for well-formedness before
any further processing is done.

\subsection{Schema's}
A schema describes the structure of the document and specifies
where each element is allowed to occur. In addition it contains
documentation. The \apex schema's are located under
\filename{/schemas} in the main \apex directory.

A document is called \emph {valid} when it adheres to the
associated schema and if the document complies with the
constraints expressed in it \footnote{a document can only be valid
if it is well-formed}. If the document is valid the experiment
file should run directly after having opened it with \apex
(although errors that are specific to \apex can still occur).
\apex will generate the necessary comments and errors if the
experiment file is not valid \footnote{by means of message
window}.

\oxygen{ For autocompletion and display of documentation it is
necessary to work in OxygenXML. You need to associate the
experiment file with the main schema, i.e. apex-schema.xsd. Click
on ``associate schema'' in Oxygen (pushpin icon) and browse for
apex-schema.xsd, which is usually located under \filename{schemas}
in the main \apex directory\footnote{The \apex schema can be found
in the \apex directory structure under
\texttt{schemas/apex-schema.xsd}}. Click on OK. Once an XML-file
is associated with the schema, you can check whether your document
is valid (menu bar: blue ``v'' icon, bottom row, second from left)
and well-formed (menu bar: red ``v'' icon, bottom row, left). }

Try the following:

\begin{itemize}

\item open the experiment file
\filename{examples/closedsetword.apx} using \apex. Run the
experiment if you wish.

\item open with OxygenXML

\item associate the \apex schema

\item check validity

\item make the file NOT well-formed, e.g. by removing an end tag.
Check validity again.

\item fix it again

\item make the file invalid, e.g. by changing the name of an
element

\item run with \apex. Read error messages.

\item undo the errors

\item check well-formedness

\item check validity



\end{itemize}


\label{sec:Schema's}

\index{well-formedness}

\index{validity}

\index{Associating File in XML editor}
