\documentclass[a4paper,12pt]{article}

\usepackage{a4wide}

\usepackage[dutch]{babel}
\usepackage[latin1]{inputenc}
\selectlanguage{dutch}

% kleinere lists
\usepackage[normalsections,,normalmargins,normalfloats,normalindent,normalleading,normaltitle,normalbib,normalbibnotes]{savetrees} 

%opening
\title{Apex - overzicht autocorrectie zinnen \footnote{Overgenomen uit Rapport 2}}
\author{Tom Francart}

\setlength{\itemsep}{0pt}

\begin{document}

\maketitle

\tableofcontents


APEX werd uitgebreid met een nieuw soort experiment: het adaptief open
set experiment. Dit experiment is gebaseerd op het bestaande open set
experiment, maar hier wordt de zin ook gecorrigeerd en wordt gebruik
gemaakt van een adaptieve procedure.

Om een zin te corrigeren wordt gebruik gemaakt van een lijst van
sleutelwoorden. Opdat een zin juist gerekend zou worden, moet aan de
volgende criteria voldaan zijn:

\begin{enumerate}
\item Elk woord uit de lijst van sleutelwoorden moet in het antwoord voorkomen
\item De woorden moeten in de juiste volgorde voorkomen
\item Een woord wordt beschouwd als aanwezig als de
  DoubleMetaphone\footnote{De DoubleMetaphone voorstelling van een
    woord is een soort fonetische voorstelling die ontwikkeld is met
    het oog op automatisch matchen van familienamen. In onze
    implementatie worden enkel de eerste 4 karakters van deze
    voorstelling in rekening gebracht. }
  representatie identiek is en de Levenshtein afstand\footnote{De
    Levenshtein afstand wordt gedefinieerd als de som van het aantal
    \begin{itemize}
\item transposities
\item verwijderingen
\item toevoegingen
\end{itemize}
Nodig om de ene string in de andere te transformeren
} kleiner is dan
  een bepaalde drempel (nu 3), dit om eventuele spelfouten niet te
  bestraffen. Deze drempel zou afhankelijk gemaakt kunnen worden van
  de lengte van het woord, maar dit wordt momenteel opgevangen door
  het gebruik van DoubleMetaphone. 
\item Leestekens worden genegeerd, alsook speciale karakters (zoals �,
  �, etc.)
\item Getallen moeten numeriek ingevoerd worden en volledig juist zijn.

\end{enumerate}

De adaptieve procedure is als volgt opgevat: De zinnen worden telkens
in dezelfde volgorde aangeboden. De eerste zin wordt eerst aangeboden
op een zeer lage SNR en telkens opnieuw aangeboden tot een correct
antwoord gegeven wordt. Indien geen correct antwoord gegeven wordt bij
de ``gemakkelijkse'' SNR, gaat de procedure gewoon verder in de hoop
dat er beter gescoord zal worden op de volgende zinnen. De rest van de
procedure is de standaard Levitt adaptieve procedure. 

Uiteindelijk wordt het 50\% verstaanbaarheidspunt berekend als het
rekenkundig gemiddelde van de $n$ laatste SNR's, waarbij standaard
$n=4$. Ook wordt het gemiddelde van de laatste $r$ reversals berekend.

~

Het voordeel van dit systeem is dat experimenten uitgevoerd kunnen
worden zonder aanwezigheid van een audioloog. Wel moet onderzocht
worden wat de statistische invloed van dit systeem is op de bekomen
resultaten. 

Om op een gebruiksvriendelijke manier sleutelwoorden aan te duiden in
een zin, werd een eenvoudige matlab interface ontwikkeld voor dit
doel en werd deze gebruikt om in alle zinnen in de ``Wivine'' lijst
deze sleutelwoorden aan te duiden. 

Bij het gebruik van dit soort experiment moeten duidelijke instructies
gegeven worden aan de gebruiker: getallen moeten numeriek ingevoerd
worden, er moeten zo min mogelijk spelfouten gemaakt worden en vooral
de spati�ring tussen woorden mag niet fout zijn. 

~

Tevens werden matlab scripts ontwikkeld om automatisch een
experiment-file te genereren op basis van de woordenlijsten.




%\bibliographystyle{plain} 
%\bibliography{/home/tom/bib/total.bib}



\end{document}
