\subsection{calibration}
\subsubsection{calibration/calibration.apx}
\begin{description}
\item[Short] 
 Simple calibration use
\item[Description] 
 The experiment shows how to calibrate using the calibration screen. You can calibrate the left and right channel seperately.
\item[How] 
 Use the calibration element with a calibration profile and an id for each channel (left and right)
\end{description}

\subsubsection{calibration/calibration-automatic-bk2250.apx}
\begin{description}
\item[Short] 
 example of automatic calibration with B\&K SLM 2250 plugin
\item[Description] 
 calibrate left and right ear at the start of the experiment
\item[How] 
 The \textless{}soundlevelmeter\textgreater{} block specifies how to use the sound level meter. If the sound level meter is not found, the regular calibration dialog is shown.
\end{description}

\subsubsection{calibration/calibration-automatic-dummyslm.apx}
\begin{description}
\item[Short] 
 example of automatic calibration with dummy SLM plugin
\item[Description] 
 Allow to experiment with automatic calibration without having a sound level meter handy
\item[How] 
 The \textless{}soundlevelmeter\textgreater{} block specifies how to use the sound level meter.
\end{description}

\subsubsection{calibration/calibration-automatic-localisation.apx}
\begin{description}
\item[Short] 
 Automatic calibration using interface to sound level meter
\item[Description] 
 13-loudspeaker sound source localisation experiment, with automatic calibration through the interface to the B\&K SLM2250 Sound Level Meter
\item[How] 
 Each channel of the sound card has a gain. These gains are specified in the \textless{}calibration\textgreater{} element. Here the \textless{}soundlevelmeter\textgreater{} block specifies how to use the sound level meter. If the sound level meter is not found, the regular calibration dialog is shown.
\end{description}

\subsubsection{calibration/calibration-connections.apx}
\begin{description}
\item[Short] 

\item[Description] 

\item[How] 

\end{description}

\subsection{childmode}
\subsubsection{childmode/childmode-car-full.apx}
\begin{description}
\item[Short] 
 Shows use of child mode: intro and outro movies, child panel
\item[Description] 
 The experiment starts after a silent introductory movie. The child needs to find the animal in one of the three cars. One car has a different sound than the two other cars.
\item[How] 
 Flash elements are introduced that read movie files with extension .swf
\end{description}

\subsubsection{childmode/childmode-extramovies.apx}
\begin{description}
\item[Short] 
 Demonstration of extra movies (snail movie)
\item[Description] 
 The experiment starts after a silent introductory movie. The child needs to find the stimulus in one of the three snails. One snail has a different sound than the two other snails. Childfriendly feedback is provided.
\item[How] 
 Flash elements of snails
\end{description}

\subsubsection{childmode/childmode-full.apx}
\begin{description}
\item[Short] 
 Shows use of child mode: intro and outro movies, child panel, progressbar, shortcuts.
\item[Description] 
 The experiment starts after a silent introductory movie. The child needs to find the stimulus in one of the three eggs. One egg has a different sound than the two other eggs. The progressbar and feedback are also childfriendly.
\item[How] 
 Flash elements are introduced that read movie files with extension.swf, child panel activated by \textless{}panel\textgreater{} in \textless{}childmode\textgreater{} (\textless{}screens\textgreater{}) + button shortcuts are introduced (button 1 = 1, button 2 = 2, button 3 = 3)
\end{description}

\subsubsection{childmode/childmode-full-L34.apx}
\begin{description}
\item[Short] 
 Shows use of child mode: intro and outro movies, child panel, progressbar for L34
\item[Description] 
 The experiment starts after a silent introductory movie. The child needs to find the stimulus in one of the three eggs. One egg has a different sound than the two other eggs.  The progressbar and feedback are also childfriendly.
\item[How] 
 Flash elements, L34 implementation in \textless{}devices\textgreater{}
\end{description}

\subsubsection{childmode/childmode-justmovies.apx}
\begin{description}
\item[Short] 
 Shows use of child mode: without intro and outro movies, but with the normal panel
\item[Description] 
 The child needs to find the stimulus in one of the three eggs. One egg has a different sound than the two other eggs. The same progressbar is used as in adult experiments, feedback is childfriendly.
\item[How] 
 Flash elements, same panel as adult experiments (no childmode selected in \textless{}screens\textgreater{})
\end{description}

\subsubsection{childmode/childmode-movies+intro.apx}
\begin{description}
\item[Short] 
 Shows use of flash movies, but without the actual childmode (no childpanel)
\item[Description] 
 The experiment starts after a silent introductory movie. The child needs to find the stimulus in one of the three eggs. The same progressbar is used as in adult experiments. Feedback is childfriendly.
\item[How] 
 flash elements, same panel as adult experiments (no \textless{}panel\textgreater{}reinforcement.swf\textless{}/panel\textgreater{} in childmode)
\end{description}

\subsubsection{childmode/childmode-nofeedback.apx}
\begin{description}
\item[Short] 
 Shows use of child mode without special feedback movies
\item[Description] 
 The experiment starts after a silent introductory movie. The child needs to find the stimulus in one of the three eggs. One egg has a different sound than the two other eggs.
\item[How] 
 Flash elements, no special feedback with flash elements provided
\end{description}

\subsubsection{childmode/childmode-noprogress.apx}
\begin{description}
\item[Short] 
 Shows use of child mode: intro and outro movies, but without the actual childmode (no childpanel)
\item[Description] 
 The experiment starts after a silent introductory movie. The child needs to find the stimulus in one of the three eggs. One egg has a different sound than the two other eggs. Feedback is childfriendly.
\item[How] 
 Flash elements, no panel or progressbar (\textless{}showpanel\textgreater{}false\textless{}/showpanel\textgreater{} and \textless{}progressbar\textgreater{}false\textless{}/progressbar\textgreater{})
\end{description}

\subsubsection{childmode/childmode-onlyscreen-normalpanel.apx}
\begin{description}
\item[Short] 
 Shows use of child mode: intro and outro movies, but with the normal panel
\item[Description] 
 The experiment starts after a silent introductory movie. The child needs to find the stimulus in one of the three eggs. One egg has a different sound than the two other eggs. The same progressbar is used as in adult experiments. Feedback is childfriendly.
\item[How] 
 Flash elements, same panel as adult experiments (no childmode selected in \textless{}screens\textgreater{})
\end{description}

\subsubsection{childmode/childmode-poll.apx}
\begin{description}
\item[Short] 
  Shows use of child mode: instead of waiting a specific time, 0 is specified as length for intro, outro and feedback, which makes apex wait for the movies to finish.
\item[Description] 
 The experiment starts after a silent introductory movie. The child needs to find the stimulus in one of the three eggs. One egg has a different sound than the two other eggs. The same progressbar is used as in adult experiments.
\item[How] 
 Flash elements, same panel as adult experiments (no childmode selected in \textless{}screens\textgreater{}), length of intro and outro is specified in \textless{}childmode\textgreater{}
\end{description}

\subsubsection{childmode/childmode-shortcuts.apx}
\begin{description}
\item[Short] 
 Shows use of child mode with shortcut keys
\item[Description] 
 The experiment starts after a silent introductory movie. The child needs to find the stimulus in one of the three eggs. One egg has a different sound than the two other eggs. The progressbar and feedback are also childfriendly.
\item[How] 
 Flash elements, button shortcuts are introduced (leftarrow = 1, downarrow = 2, rightarrow = 3)
\end{description}

\subsubsection{childmode/childmode-skipintrooutro.apx}
\begin{description}
\item[Short] 
 Shows use of child mode: intro and outro movies, but without the child panel; also has button shortcuts to skip the intro or outro movie
\item[Description] 
 The experiment starts after a silent introductory movie. The child needs to find the stimulus in one of the three eggs. One egg has a different sound than the two other eggs. Feedback is also childfriendly.
\item[How] 
 The F7 shortcut can be used to skip the intro and outro movies when \textless{}allowskip\textgreater{}true\textless{}/allowskip\textgreater{} in \textless{}general\textgreater{}. Flash elements, button shortcuts are introduced (leftarrow = 1, downarrow = 2, rightarrow = 3, 's' = skipping intro or outro movie),
\end{description}

\subsubsection{childmode/childmode-skipintrooutro-disabled.apx}
\begin{description}
\item[Short] 
 Skip intro or outro movie by hitting 's' on the keyboard is disabled.
\item[Description] 
 The experiment starts after a silent introductory movie. The child needs to find the stimulus in one of the three eggs. One egg has a different sound than the two other eggs. Feedback is childfriendly.
\item[How] 
 Flash elements, skipping is disabled by setting  \textless{}allowskip\textgreater{}false\textless{}/allowskip\textgreater{} in \textless{}general\textgreater{}
\end{description}

\subsubsection{childmode/childmode-snake-full.apx}
\begin{description}
\item[Short] 
 Shows use of child mode: intro and outro movies, child panel, progressbar
\item[Description] 
 The experiment starts after a silent introductory movie. The child needs to find the stimulus in one of the three snakes. One snake has a different sound than the two other snakes. The progressbar and feedback are also childfriendly.
\item[How] 
 Flash elements of snakes, child panel activated by \textless{}panel\textgreater{} in \textless{}childmode\textgreater{} (\textless{}screens\textgreater{})
\end{description}

\subsubsection{childmode/childmode-waitforstart.apx}
\begin{description}
\item[Short] 
 Shows use of child mode, with no shortcut keys and waiting for start before every trial
\item[Description] 
 The experiment starts after a silent introductory movie. The child needs to find the stimulus in one of the three eggs. One egg has a different sound than the two other eggs. The next trial is only presented after selecting Start from the Experiment menu or pressing F5.
\item[How] 
 Flash elements + \textless{}waitforstart\textgreater{} function in \textless{}general\textgreater{}
\end{description}

\subsection{connections}
\subsubsection{connections/connections-all-mixed-monostereo.apx}
\begin{description}
\item[Short] 
 The use of ALL connections
\item[Description] 
 Monaural presentations of digit 1. Level is changed by clicking the buttons quieter and louder
\item[How] 
 Connections defines how the different datablocks and filters are routed to the output device
\end{description}

\subsubsection{connections/connections-byname.apx}
\begin{description}
\item[Short] 
 Try to autoconnect a stereo wavfile to a mono wavdevice by name
\item[Description] 
 Digits are presented dichotically, you can give a verbal response
\item[How] 
 Connections are used to route different datablocks and filters to an output device
\end{description}

\subsubsection{connections/connections-byregexp.apx}
\begin{description}
\item[Short] 
 Try to autoconnect a stereo wavfile to a mono wavdevice using mode=regexp
\item[Description] 
 Dichotic presentation of digits, you can give a verbal response
\item[How] 
 Connections are used to route different datablocks and filters to an output device
\end{description}

\subsubsection{connections/connections-bywildcard.apx}
\begin{description}
\item[Short] 
 Try to autoconnect a stereo wavfile to a mono wavdevice using mode=wildcard
\item[Description] 
 Dichotic presentation of digits, you can give a verbal response
\item[How] 
 Connections are used to route different datablocks and filters to an output device
\end{description}

\subsubsection{connections/connections-negativechannel.apx}
\begin{description}
\item[Short] 
 Test the use of channel -1
\item[Description] 
 If something is connected to channel -1 of the output device, it should be ignored, no output when button wrong is clicked
\item[How] 
 Connections are used to route filters to output devices
\end{description}

\subsection{controller}
\subsubsection{controller/controller-plugincontroller.apx}
\begin{description}
\item[Short] 
 Plugincontroller
\item[Description] 
 Gain of plugin controller varies according to user input
\item[How] 
 the democontroller is used here to demonstrate how a plugin controller can be implemented.
\end{description}

\subsubsection{controller/osccontroller.apx}
\begin{description}
\item[Short] 
\item[Description] 
\item[How] 
\end{description}

\subsubsection{controller/osccontroller\_avatar.apx}
\begin{description}
\item[Short] 
\item[Description] 
\item[How] 
\end{description}

\subsection{datablocks}
\subsubsection{datablocks/datablocks-combination.apx}
\begin{description}
\item[Short] 
 Combination of datablocks using sequential and simultaneous
\item[Description] 
  Play datablocks similtaneously and sequentially
\item[How] 
 Using the \textless{}simultaneous\textgreater{} and \textless{}sequential\textgreater{} tags in \textless{}stimulus\textgreater{}, datablocks can be organised and reused.
\end{description}

\subsubsection{datablocks/datablocks-loop.apx}
\begin{description}
\item[Short] 
 Use the same datablock multiple times in a stimulus
\item[Description] 
 The datablock wavfile "één" is played twice within one stimulus
\item[How] 
 set the number of replications of the datablock by setting the number of the loop, here it is \textless{}loop\textgreater{}2\textless{}/loop\textgreater{}
\end{description}

\subsubsection{datablocks/datablocks-nodevice.apx}
\begin{description}
\item[Short] 
 demonstration of omission of device in datablock
\item[Description] 
 When there is only one device in the experiment, it does not need to be explicitly mentioned in each datablock.
\item[How] 
 APEX will automatically assign the device to each datatblock
\end{description}

\subsubsection{datablocks/datablocks-repeat.apx}
\begin{description}
\item[Short] 
 Use the same datablock multiple times in a stimulus
\item[Description] 
 The datablock wavfile "één" is played four times within one stimulus
\item[How] 
 sequential datablocks within stimulus1
\end{description}

\subsubsection{datablocks/datablocks-silence.apx}
\begin{description}
\item[Short] 
 Combination of silence and signal for 1 stimulus presentation
\item[Description] 
 The stimulus in this example contains first silence (a delay) before you hear "één"
\item[How] 
 Use sequential datablocks that include silence and wavfile/signals, e.g. also possible to put silence datablocks after the wavfile
\end{description}

\subsubsection{datablocks/datablocks-simultaneous.apx}
\begin{description}
\item[Short] 
 Play datablocks simultaneously
\item[Description] 
 Demonstrate the use of simultaneous datablocks in a stimulus
\item[How] 
 When datablocks are listed in the \textless{}simultaneous\textgreater{} element in \textless{}stimulu\textgreater{} they are played simultaneously
\end{description}

\subsection{filters}
\subsubsection{filters/filters-amplifier-defaultparameters.apx}
\begin{description}
\item[Short] 
 Show amplifier default parameters
\item[Description] 
 the stimulus wd1.wav is played in the left ear, when you click the button 'play with default parameters', the stimulus is played again with the same gain (=0) (10 trials)
\item[How] 
 fixed\_parameters
\end{description}

\subsubsection{filters/filters-amplifier-setgain.apx}
\begin{description}
\item[Short] 
 Adapting device parameters: gain of device varies according to user input
\item[Description] 
 the stimulus wd1.wav and noise are played simultaneously in the left ear, when you click the button 'higher'/'lower', the stimulus and the noise are played again with a different gain for the noise (= gain + 2/- 2, see output box 'parameterlist')
\item[How] 
 adaptiveProcedure: gain is adjusted with stepsize = 2
\end{description}

\subsubsection{filters/filters-complexprocessing.apx}
\begin{description}
\item[Short] 
 More complex processing, can be used for testing skips - 2 noise stimuli (sinus \& wivineruis.wav) with different gains (filters)
\item[Description] 
 When you click on 'wrong' you hear in both ears two noise stimuli and in the left ear 2 times a sentence, the same happens when you click on the 'wrong' button again
\item[How] 
 trainingProcedure, creating a sinus as noise-stimulus
\end{description}

\subsubsection{filters/filters-dataloop-2simultaneous.apx}
\begin{description}
\item[Short] 
 Shows use of dataloop generator : uses the same datablock twice (for 2 different dataloops)
\item[Description] 
 When you click on the button 'correct/wrong', in both ears a noise stimulus is played and in the left ear 'one-two'(=stimulus1)/'silence-two'(=stimulus2) is presented - Dataloop: playing noise and according to the users input (correct/wrong) stimulus1 or stimulus2 is presented
\item[How] 
 2 dataloops: noise for trial1 and trial2
\end{description}

\subsubsection{filters/filters-dataloop.apx}
\begin{description}
\item[Short] 
 Shows use of dataloop generator - wivineruis.wav should play !not! continuously over trials (see filters - noisegen - apex:dataloop)
\item[Description] 
 When you click on the button 'wrong/correct', in both ears a noise stimulus is played and in the left ear 'silence-two(twee)'(=stimulus2)/'one-two(een-twee)'(=stimulus1) is presented
\item[How] 
 trainingProcedure - dataloop generator - continuous: false
\end{description}

\subsubsection{filters/filters-dataloop-continuous.apx}
\begin{description}
\item[Short] 
 Shows use of dataloop generator - wivineruis.wav should play continuously over trials (see filters - noisegen - apex:dataloop)
\item[Description] 
 The noise is playing continuously over trials. When you click on the button 'dataloop - silence/dataloop - 1', in both ears a noise stimulus is played and in the left ear 'silence-two(twee)'(=stimulus2)/'one-two(een-twee)'(=stimulus1) is presented
\item[How] 
 trainingProcedure - dataloop generator - continuous: true
\end{description}

\subsubsection{filters/filters-dataloop-jump.apx}
\begin{description}
\item[Short] 
 Shows use of jump parameter of dataloop generator
\item[Description] 

\item[How] 

\end{description}

\subsubsection{filters/filters-dataloop-randomjump.apx}
\begin{description}
\item[Short] 

\item[Description] 

\item[How] 

\end{description}

\subsubsection{filters/filters-noisegen-setgain.apx}
\begin{description}
\item[Short] 
 Basegain of noise: 0 - Aapting device parameters - gain of device varies according to user input
\item[Description] 
 Noise and a stimulus (one-een) are presented simultaneously in the left ear, when you click on 'higher/lower', you hear again the noise and stimulus in the left ear but the gain of the noise is adjusted with +2/-2 (=stepsize, see output box)
\item[How] 
 adaptiveProcedure - stepsize of gain - adapt\_parameter (Procedure) - continuous:true
\end{description}

\subsubsection{filters/filters-plugin.apx}
\begin{description}
\item[Short] 

\item[Description] 
 example already somewhere else as well!
\item[How] 

\end{description}

\subsubsection{filters/filters-plugin-matlabfilter.apx}
\begin{description}
\item[Short] 
 Example using the matlabfilter plugin.
\item[Description] 
  The plugin will use matlab to process data, one buffer at a time. Note that you should make sure your system can find the Matlab eng library.
\item[How] 
 \textless{}plugin\textgreater{}matlabfilter\textless{}/plugin\textgreater{} in \textless{}filters\textgreater{}.
\end{description}

\subsubsection{filters/filters-plugin-scamblespectrumfilter.apx}
\begin{description}
\item[Short] 
 Demonstrate scramblespectrum filter
\item[Description] 
 The scramblespectrum filter will randomize the spectrum to reduce monaural spectral cues in a localisation experiment
\item[How] 
 parameters to the scramblespectrumfilter are specified in \textless{}filter\textgreater{}
\end{description}

\subsubsection{filters/filters-plugin-scamblespectrumfilter\_calibration.apx}
\begin{description}
\item[Short] 

\item[Description] 

\item[How] 

\end{description}

\subsubsection{filters/filters-pulsegen.apx}
\begin{description}
\item[Short] 
        Shows synchronisation for L34 using a soundcard pulse
\item[Description] 
 When you click on send, a pulse is presented in your left ear
\item[How] 
 filters: generator makes a soundcard pulse =\textgreater{} 2 channels of stereo file are mixed into output ch 0, the pulse goes to ch 1 (left ear)
\end{description}

\subsubsection{filters/filters-sinegenerator.apx}
\begin{description}
\item[Short] 
 Test of setting frequency parameter of sine generator - frequency of sinus varies according to user input
\item[Description] 
 A sinus is playing in the left ear when you click on one of the 3 buttons. If you click on 100Hz/1000Hz/defaults, the frequency of the sinus changes to 100Hz/1000Hz/10 000Hz.
\item[How] 
 filter: 1 generator: sinus (10 000Hz) - 3 stimuli: 3x silence with adjusted parameters (frequency: 100Hz/1000Hz/no change) -  continuous: false
\end{description}

\subsubsection{filters/filters-sinegen-setfrequency.apx}
\begin{description}
\item[Short] 
 Regression test for adapting device parameters - gain of device varies according to user input
\item[Description] 
 A sinus is played in the left ear (sinus is played continuously over trials), when you click the button 'higher'/'lower', the sinus is played again with a different frequency for the sinus (= frequency + 100Hz/- 100Hz, see output box 'parameterlist')
\item[How] 
 adaptiveProcedure: frequency (adapt\_parameter) is adjusted with stepsize = 100Hz - datablock/stimulus: silence
\end{description}

\subsubsection{filters/filters-sinegen-setgain.apx}
\begin{description}
\item[Short] 
 Regression test for adapting device parameters - gain of device varies according to user input
\item[Description] 
 A sinus is played in the left ear. The sinus is played continuously over trial but when you click on 'higher/lower' the sinus is played again but the gain of the sinus is adapted with +5/-5 dB.
\item[How] 
 adaptiveProcedure - datablock/stimulus: silence - filter: generator: sinus - continuous: true - adapt\_parameter: gain - (!basegain has to be low enough otherwise there is clipping!)
\end{description}

\subsubsection{filters/filters-singlepulse-defaultparameters.apx}
\begin{description}
\item[Short] 
        Shows synchronisation  using a soundcard pulse
\item[Description] 
 When you click on 'play with default parameters', a pulse is presented in your left ear simultaneously with a stimulus (wd1.wav - een/one)
\item[How] 
 datablock/stimulus + filter: generator makes a soundcard pulse (negative polarity)
\end{description}

\subsubsection{filters/filters-singlepulse-polarity.apx}
\begin{description}
\item[Short] 
        Shows synchronisation for L34 using a soundcard pulse with a negative polarity (filter - pulsegen)
\item[Description] 
 When you click on send, a pulse is presented in your left ear and a stimulus (datablock: \_Input44\_2.wav) is presented in your right ear
\item[How] 
 filters: generator makes a soundcard pulse with a negative polarity
\end{description}

\subsubsection{filters/filters-vocoder.apx}
\begin{description}
\item[Short] 
 Demonstration of vocoder plugin.
\item[Description] 
 Play a sentence either vocoded or unvocoded
\item[How] 
 A vocoder-plugin filter has been added in the connections.
\end{description}

\subsubsection{filters/filters-wavamplifier.apx}
\begin{description}
\item[Short] 
 Regression test for training procedure - Output: number according to last input
\item[Description] 
 When you click on the button 'play': in the left ear/channel: "een", is presented and in the right ear/channel: "twee", 20dB louder compared to the left ear, is presented.
\item[How] 
 filter: amplifier =\textgreater{} basegain (-10) + gain left channel(-10)/right channel(+10)
\end{description}

\subsubsection{filters/filters-wavfadeincos.apx}
\begin{description}
\item[Short] 
 Regression test for fader (begin/end of stimulus is not abrupt)
\item[Description] 
 When you click on the button 'play fade/play NO fade', in both ears a sinus is presented with at the beginning of the stimulus a fade-in/NO fade-in (click)
\item[How] 
 filter: fader in begin (cosine) =\textgreater{} length: 400ms, direction: in (fade-in: begin of stimulus)
\end{description}

\subsubsection{filters/filters-wavfadeinlin.apx}
\begin{description}
\item[Short] 
 Regression test for fader (begin/end of stimulus is not abrupt)
\item[Description] 
 When you click on the button play fade/play NO fade', in both ears a sinus is presented with at the beginning of the stimulus a fade-in/NO fade-in (click)
\item[How] 
 filter: fader(linear) =\textgreater{} length: 400ms, direction: in (fade-in: begin of stimulus)
\end{description}

\subsubsection{filters/filters-wavfadeoutcos.apx}
\begin{description}
\item[Short] 
 Regression test for fader (begin/end of stimulus is not abrupt)
\item[Description] 
 When you click on the button 'play fade/play NO fade', in both ears a sinus is presented with at the end of the stimulus a fade-out/NO fade-in (click)
\item[How] 
 filter: fader(cosine) =\textgreater{} length: 400ms, direction: out (fade-out: end of stimulus)
\end{description}

\subsubsection{filters/filters-wavfadeoutlin.apx}
\begin{description}
\item[Short] 
 Regression test for fader (begin/end of stimulus is not abrupt)
\item[Description] 
 When you click on the button 'play fade/play NO fade', in both ears a sinus is presented with at the end of the stimulus a fade-out/NO fade-in (click)
\item[How] 
 filter: fader(linear) =\textgreater{} length: 400ms, direction: out (fade-out: end of stimulus)
\end{description}

\subsection{general}
\subsubsection{general/general-autosave.apx}
\begin{description}
\item[Short] 
 Automatically save results to apr-file after experiment
\item[Description] 
 Closed-set identification task with automatic saving of results
\item[How] 
 General - autosave
\end{description}

\subsubsection{general/general-exitafter.apx}
\begin{description}
\item[Short] 
 Exit experiment when finished
\item[Description] 
 Closed-set identification task, exit after finish
\item[How] 
 General - exitafter=true
\end{description}

\subsubsection{general/general-prefix-absolutepath.apx}
\begin{description}
\item[Short] 
 Get the datablock prefix from the absolute path
\item[Description] 
 Closed-set identification task
\item[How] 
 datablocks uri\_prefix full path
\end{description}

\subsubsection{general/general-prefix-frommainconfig.apx}
\begin{description}
\item[Short] 
   Get the datablock prefix from the main configfile
\item[Description] 
 Closed-set identification task
\item[How] 
 source=apexconfig
\end{description}

\subsubsection{general/general-prefix-invalid.apx}
\begin{description}
\item[Short] 
 Get the datablock prefix from the main configfile
\item[Description] 
 Invalid prefix id specified: wavfile not found wd1.wav
\item[How] 
 source=apexconfig
\end{description}

\subsubsection{general/general-scriptparameters.apx}
\begin{description}
\item[Short] 
 Demonstrate the use of general/scriptparameters
\item[Description] 
 Digits 1-7 are presented monaurally, you can click on a button to go to the next one
\item[How] 
 Scriptparameters - name=path, scriptparameters.js needed
\end{description}

\subsubsection{general/general-showresults.apx}
\begin{description}
\item[Short] 
 Show results after experiment
\item[Description] 
 When the experiment is finished and the results are saved, you can choose to see the results
\item[How] 
 Results - showafterexperiment=true
\end{description}

\subsubsection{general/general-xmlerror.apx}
\begin{description}
\item[Short] 
 Error in XML file
\item[Description] 

\item[How] 

\end{description}

\subsection{interactive}
\subsubsection{interactive/invalid-entry.apx}
\begin{description}
\item[Short] 
 Change small parameter right before the start of the experiment
\item[Description] 
 GUI is shown with the element to be changed
\item[How] 
 Interactive
\end{description}

\subsubsection{interactive/setgain.apx}
\begin{description}
\item[Short] 
 GUI to change some settings right before the experiment
\item[Description] 
 GUI appears to set/change some settings
\item[How] 
 Interactive
\end{description}

\subsubsection{interactive/subject.apx}
\begin{description}
\item[Short] 
 GUI to change some settings right before the experiment
\item[Description] 
 GUI appears to set/change some settings
\item[How] 
 The name of the subject, in \textless{}results\textgreater{}/\textless{}subject\textgreater{} is set from the interactive dialog that appears before the experiment is loaded.
\end{description}

\subsection{l34}
\subsubsection{l34/addinvalidfilter.apx}
\begin{description}
\item[Short] 
 Try to add wav filter to L34Device
\item[Description] 
 Since it is not possible to add a wav filter to an L34Device, an error occurs
\item[How] 
 see filters and device
\end{description}

\subsubsection{l34/bilateral.apx}
\begin{description}
\item[Short] 
 Test of synchronized bilateral CI setup
\item[Description] 
 Test whether the stimuli are given from the first, the second or both L34 devices
\item[How] 
 Including two L34 devices, stimuli can be presented stimultaneously
\end{description}

\subsubsection{l34/bilateral-singlepulse.apx}
\begin{description}
\item[Short] 
 Test of synchronized bilateral CI setup with pulse stimuli
\item[Description] 
 Test whether the stimuli are given from the first, the second or both L34 devices
\item[How] 
 Including two L34 devices, stimuli can be presented stimultaneously
\end{description}

\subsubsection{l34/invaliddatablock.apx}
\begin{description}
\item[Short] 
 Try to parse invalid datablock, should return error
\item[Description] 
 datablock does not exist in folder stimuli, running the experiment gives an error.
\item[How] 
 uri\_prefix is used to find stimuli, invalid.aseq does not exist
\end{description}

\subsubsection{l34/L34DummySync.apx}
\begin{description}
\item[Short] 
        Shows synchronisation for L34 using a soundcard pulse
\item[Description] 
         2 channels of stereo file are mixed into output ch 0, the pulse goes to ch 1
\item[How] 
 use dummyDeviceType and wavDeviceType and check connections
\end{description}

\subsubsection{l34/L34Sync.apx}
\begin{description}
\item[Short] 
        Shows synchronisation for L34 using a soundcard pulse
\item[Description] 
         2 channels of stereo file are mixed into output ch 0, the pulse goes to ch 1
\item[How] 
 use dummyDeviceType and wavDeviceType and check connections
\end{description}

\subsubsection{l34/loudness\_balancing.apx}
\begin{description}
\item[Short] 
 Loudness balancing between two stimulation CI electrodes
\item[Description] 
 Two signals are presented consecutively in the same CI, at two different stimulation electrodes. One electrode is the reference electrode and has a fixed current level, the comparison electrode has changing levels. The participant has to judge which signal is louder. An adaptive procedure is used to determine the balanced level.
\item[How] 
 Use of adaptiveProcedure, plugindatablocks, balancing.js needed
\end{description}

\subsubsection{l34/mapping1.apx}
\begin{description}
\item[Short] 
 Use default map with current units between 1 and 255
\item[Description] 

\item[How] 
 See defaultmap
\end{description}

\subsubsection{l34/mapping2.apx}
\begin{description}
\item[Short] 
 Use real map to check mapping
\item[Description] 

\item[How] 
 See defaultmap values
\end{description}

\subsubsection{l34/r126wizard.apx}
\begin{description}
\item[Short] 
 R126 is the clinical fitting software
\item[Description] 

\item[How] 
 Use \textless{}defaultmap\textgreater{} \textless{}from\_r126/\textgreater{}
\end{description}

\subsubsection{l34/rfgenxs-bilateral.apx}
\begin{description}
\item[Short] 
 Test of synchronized bilateral CI setup with an RFGenXS
\item[Description] 
 Test whether the stimuli are given on the first, the second or both channels
\item[How] 
 With an RFGenXS, stimuli can be presented stimultaneously
\end{description}

\subsubsection{l34/setvolume.apx}
\begin{description}
\item[Short] 
 Change volume of CI stimuli (send simple XML file to L34 device)
\item[Description] 
 Change the volume of the CI stimuli to 10,50,70 or 100 current units
\item[How] 
 Add variable parameter id=l34volume, see device: volume 100 is default
\end{description}

\subsubsection{l34/simpleaseq.apx}
\begin{description}
\item[Short] 

\item[Description] 

\item[How] 

\end{description}

\subsubsection{l34/simpleaseq-dummy.apx}
\begin{description}
\item[Short] 

\item[Description] 

\item[How] 

\end{description}

\subsubsection{l34/simpleaseq-examples.apx}
\begin{description}
\item[Short] 
 power-up test?
\item[Description] 

\item[How] 

\end{description}

\subsubsection{l34/simpleaseq-sp12.apx}
\begin{description}
\item[Short] 

\item[Description] 

\item[How] 

\end{description}

\subsubsection{l34/triggertest.apx}
\begin{description}
\item[Short] 

\item[Description] 

\item[How] 

\end{description}

\subsubsection{l34/wordsaseq.apx}
\begin{description}
\item[Short] 

\item[Description] 

\item[How] 

\end{description}

\subsection{manual}
\subsubsection{manual/closedsetword.apx}
\begin{description}
\item[Short] 
 Closed-set identification of words in noise with figures
\item[Description] 
 A word is presented in noise and the subject responds by clicking on one of the 4 figures on the screen, repeated 3 times. Initial SNR is set via interactive GUI.
\item[How] 
 full experiment
\end{description}

\subsubsection{manual/gapdetection.apx}
\begin{description}
\item[Short] 
 Gap detection
\item[Description] 

\item[How] 

\end{description}

\subsubsection{manual/gapdetectionchild.apx}
\begin{description}
\item[Short] 
 gap detection for children
\item[Description] 

\item[How] 

\end{description}

\subsubsection{manual/opensetidentification.apx}
\begin{description}
\item[Short] 
 Open set identification task
\item[Description] 
 A word is presented in noise and the subject responds by typing the word, repeated 3 times. Initial SNR is set via interactive GUI.
\item[How] 
 Full experiment
\end{description}

\subsubsection{manual/sentenceinnoise.apx}
\begin{description}
\item[Short] 
 Sentences in noise task
\item[Description] 
 A sentence is presented in noise, the subject responds verbally, the test leader can indicatie whether the response was correct or incorrect, repeated 4 times. Initial SNR is set via interactive GUI.
\item[How] 
 Full experiment
\end{description}

\subsubsection{manual/trainingprocedure.apx}
\begin{description}
\item[Short] 
 Training for constant procedure
\item[Description] 
 A stimulus is presented after the user had indicated the trial by pressing a button on the screen
\item[How] 
 apex:trainingProcedure
\end{description}

\subsection{parameters}
\subsubsection{parameters/parameters-connection-filter.apx}
\begin{description}
\item[Short] 
 Change the channel of a connection + stimulus has gain
\item[Description] 
 Click on the right (or left button) to hear the sound in your right ear (or left ear).
\item[How] 
 Training procedure, Fixed (stimulus) and variable parameters (channel) + possibility in code to change the gain of the stimuli with filter (amplifier)
\end{description}

\subsubsection{parameters/parameters-connection-soundcard.apx}
\begin{description}
\item[Short] 
 Change the channel of a connection using a parameter
\item[Description] 
 Click on the right (or left button) to hear the sound in your right ear (or left ear)
\item[How] 
 training procedure, Fixed (stimulus) and variable parameters (channel)
\end{description}

\subsubsection{parameters/parameters-device-allchannels.apx}
\begin{description}
\item[Short] 
 Change parameter (gain) of device channel
\item[Description] 
 gain of device varies in BOTH channels according to user input (1 button = louder, 1 button = quieter)
\item[How] 
 adaptive procedure, fixed (stimulus) and variable (gain) parameters, stepsize of 2 dB
\end{description}

\subsubsection{parameters/parameters-device-singlechannel.apx}
\begin{description}
\item[Short] 
 Change parameter (gain) of device channel
\item[Description] 
 Gain of device varies in LEFT channel according to user input (1 button = louder, 1 button = quieter)
\item[How] 
 Adaptive procedure, fixed (stimulus) and variable (gain) parameters, stepsize of 2 dB
\end{description}

\subsubsection{parameters/parameters-filter.apx}
\begin{description}
\item[Short] 
 Change parameter (gain) of device channel
\item[Description] 
 gain of device varies in BOTH channel according to user input (1 button = louder, 1 button = quieter) + gain is visible on the screen
\item[How] 
 adaptive procedure, fixed (stimulus) and variable (gain) parameters, stepsize of 2 dB, Parameterlists are introduced that show Left or Right gain on the screen
\end{description}

\subsubsection{parameters/parameters-filter-channel.apx}
\begin{description}
\item[Short] 
 Change parameter (gain) of device channel
\item[Description] 
 gain of device varies in RIGHT channel according to user input (1 button = louder, 1 button = quieter) + gain is visible on screen
\item[How] 
 adaptive procedure, fixed (stimulus) and variable (gain) parameters, stepsize of 2 dB, Parameterlists are introduced that show Left or Right gain on the screen
\end{description}

\subsubsection{parameters/parameters-filter-channel-probe.apx}
\begin{description}
\item[Short] 
  Change parameter (gain) of device channel
\item[Description] 
 gain of device varies in LEFT channel according to user input (1 button = louder, 1 button = quieter) + gain is visible on screen
\item[How] 
 adaptive procedure, fixed (stimulus) and variable (gain) parameters, stepsize of 10 dB, Parameterlists are introduced that show Left or Right gain on the screen. Additionally there is a probe filter that saves the output to disk.
\end{description}

\subsubsection{parameters/parameters-restore.apx}
\begin{description}
\item[Short] 
 Demonstrate restoring parameter values
\item[Description] 
 When a parameter is set from a stimulus, and subsequently a stimulus is played in which the parameter is not set, it should be restored to its default value
\item[How] 
 Parameter gain is set in stimulus1, and is not set in stimulus2. Therefore a gain of 0dB should be applied for stimulus2
\end{description}

\subsubsection{parameters/parameters-spinbox.apx}
\begin{description}
\item[Short] 
 Adjust noise manually while presenting speech and noise
\item[Description] 
 gain of noise varies in right channel according to user input (1 button = arrow up, 1 button = arrow down) while the stimulus stays the same.
\item[How] 
 constant procedure, Fixed (stimulus) and variable (noise gain), stepsize of 1 dB, noise is generated by filter
\end{description}

\subsubsection{parameters/parameters-wavfadeinout.apx}
\begin{description}
\item[Short] 
 Set parameters of fader
\item[Description] 
 The fader filter allows to apply an up and down ramp to a channel
\item[How] 
 There are two faders in \textless{}filters, faderin and faderout. Their parameters are set from the stimuli.
\end{description}

\subsection{procedure}
\subsubsection{procedure/adaptive-1up-1down.apx}
\begin{description}
\item[Short] 
 1up-1down Adaptive procedure: Frequency of ups-downs is the same - Experiment stops after 6 reversals
\item[Description] 
 When you click on the button 'correct or wrong', you hear a stimulus 'wd1.wav (een/one)' in both ears. When you click again, the gain of the stimulus decreases/increases according to the button (correct/false)
\item[How] 
 adaptiveProcedure - stop\_after\_type: reversals - nUp/nDown - adapt\_parameter (gain: gain of stimulus) -
\end{description}

\subsubsection{procedure/adaptive-1up-2down.apx}
\begin{description}
\item[Short] 
 1up-2down Adaptive procedure: Frequency of ups-downs is NOT the same - Experiment stops after 6 reversals
\item[Description] 
 When you click on the button 'correct or wrong', you hear a stimulus 'wd1.wav (een/one)' in both ears. When you click again, the gain of the stimulus StaysTheSame/Decreases/Increases according to the button (correct/false) and the number of ups and downs
\item[How] 
 adaptiveProcedure - stop\_after\_type: reversals - nUp/nDown - adapt\_parameter (gain: gain of stimulus)
\end{description}

\subsubsection{procedure/adaptive-2up-1down.apx}
\begin{description}
\item[Short] 
 2up-1down Adaptive procedure: Frequency of ups-downs is NOT the same - Experiment stops after 6 reversals
\item[Description] 
 When you click on the button 'correct or wrong', you hear a stimulus 'wd1.wav (een/one)' in both ears. When you click again, the gain of the stimulus StaysTheSame/Decreases/Increases according to the button (correct/false) and the number of ups and downs
\item[How] 
 adaptiveProcedure - stop\_after\_type: reversals - nUp/nDown - adapt\_parameter (gain: gain of stimulus)
\end{description}

\subsubsection{procedure/adaptive\_stopafter\_presentations.apx}
\begin{description}
\item[Short] 
 Stop after a specified number of presentations (3) - should in this case stop after 6 trials (presentation: every trial is presented once =\textgreater{} 2 trials =\textgreater{} 2x3 = 6)
\item[Description] 
 When you click on the button 'correct or wrong', you hear a stimulus 'wd1.wav (een/one)' in both ears. When you click again, the gain of the stimulus StaysTheSame/Decreases/Increases according to the button (correct/false) and the number of ups and downs
\item[How] 
 adaptiveProcedure - stop\_after\_type: presentations - adapt\_parameter (gain: gain of stimulus)
\end{description}

\subsubsection{procedure/adaptive\_stopafter\_presentations\_invalid.apx}
\begin{description}
\item[Short] 
 Error message because \#presentations is not equal to \#stop\_after
\item[Description] 
 Point of interest if you want to stop the experiment after some presentations! =\textgreater{} \#presentations = \#stop\_after
\item[How] 
 procedure: presentations - stop\_after
\end{description}

\subsubsection{procedure/adaptive\_stopafter\_reversals.apx}
\begin{description}
\item[Short] 
 Stop after a specified number of reversals (6) - changes between correct-false
\item[Description] 
 When you click on the button 'correct or wrong', you hear a stimulus 'wd1.wav (een/one)' in both ears. When you click again, the gain of the stimulus Decreases/Increases according to the button (correct/false) and the stepsizes changes after 3 reversals
\item[How] 
 adaptiveProcedure - stop\_after\_type: reversals - adapt\_parameter (gain: gain of stimulus) - stepsize: change\_after: reversals
\end{description}

\subsubsection{procedure/adaptive\_stopafter\_trials.apx}
\begin{description}
\item[Short] 
 Stop after a specified number of trials (10)
\item[Description] 
 When you click on the button 'correct or wrong', you hear a stimulus 'wd1.wav (een/one)' in the left ear. When you click again, the gain of the stimulus Decreases/Increases according to the button (correct/false)
\item[How] 
 adaptiveProcedure - stop\_after\_type: trials - adapt\_parameter (gain: gain of stimulus)
\end{description}

\subsubsection{procedure/adjustment-pluginprocedure.apx}
\begin{description}
\item[Short] 
 Adjustment of stimuli with a pluginprocedure
\item[Description] 
 There are 6 buttons:
\item[How] 
 adaptiveProcedure - stop\_after\_type: trials - adapt\_parameter (gain: gain of stimulus)
\end{description}

\subsubsection{procedure/adp1.apx}
\begin{description}
\item[Short] 
 Regression test for ADP - Experiment stops after 4 reversals
\item[Description] 
 adaptive procedure, one of the following sequences are played: noise een noise, een noise noise, noise noise een =\textgreater{} answer: one of the three sequences. Stepsize changes after 2 trials =\textgreater{} stepsize determines the parameter 'snr'
\item[How] 
 adaptiveProcedure - adapt\_parameter (snr: snr (order, not in dB) - intervals - stepsize determines the change in snr (1-2-3) not the value in dB
\end{description}

\subsubsection{procedure/adp2.apx}
\begin{description}
\item[Short] 
 Regression test for ADP - Experiment stops after 10 reversals
\item[Description] 
 adaptive procedure, one of the following sequences are played: noise een noise, een noise noise, noise noise een =\textgreater{} answer: one of the three sequences. Stepsize changes after 2 trials =\textgreater{} stepsize determines the parameter 'snr'
\item[How] 
 adaptiveProcedure - adapt\_parameter (snr: snr (order, not in dB) - intervals - stepsize determines the change in snr (1-2-3) not the value in dB
\end{description}

\subsubsection{procedure/afc-choices.apx}
\begin{description}
\item[Short] 
 Regression test - 4 Intervals - 4 Choices, select interval 2-3
\item[Description] 
 Four intervals are possible= 'noise noise noise een', 'noise noise een noise', 'noise een noise noise' and 'een noise noise noise'. However only possibility number "2" and "3" are selected
\item[How] 
 constantProcedure - intervals- count/possibilities: 4, select(ion): 2 and 3
\end{description}

\subsubsection{procedure/afc-uniquestandard.apx}
\begin{description}
\item[Short] 
 Regression test for Intervals, Standard(reference signal), Uniquestandard(true/false = If uniquestandard is true and multiple standards are defined per trial, Apex will try to present another standard in each interval of the trial)
\item[Description] 
 In each trial, the 3 standards should be used (uniquestandard = true)
\item[How] 
 constantProcedure - intervals- uniquestandard: true
\end{description}

\subsubsection{procedure/aid1.apx}
\begin{description}
\item[Short] 
 Regression test - Different stimuli can occur during one trial (according to the difficulty of the experiment ('snr') (- no connections or gain!)
\item[Description] 
 adaptive procedure, one of the following stimuli are played: 'een', 'twee', 'drie', 'vier' or 'vijf' =\textgreater{} If you click correct/false the experiment becomes more difficult/easy corresponding with the 'snr' parameter - the larger the values of snr, the easier the experiment
\item[How] 
 adaptiveProcedure - adapt\_parameter (snr: snr (order, not in dB) - different stimuli (trial) - stepsize determines the change in snr (1-2-3-4-5) - changes afther 5 trials
\end{description}

\subsubsection{procedure/aid-selectrandom.apx}
\begin{description}
\item[Short] 
 Regression test - Different stimuli can occur during one trial (according to the difficulty of the experiment ('snr')
\item[Description] 
 adaptive procedure, one of the following stimuli are played: 'een', 'twee', 'drie', 'vier' or 'vijf', each with a different gain =\textgreater{} If you click correct/false the experiment becomes more difficult/easy corresponding with the 'snr' parameter
\item[How] 
 adaptiveProcedure - adapt\_parameter (snr: snr (order, not in dB) - different stimuli (trial) - stepsize determines the change in snr (1-2-3-4-5), gain(filters)
\end{description}

\subsubsection{procedure/choices-randomstimulus-randomstandard.apx}
\begin{description}
\item[Short] 
\item[Description] 
\item[How] 
\end{description}

\subsubsection{procedure/continuous-waitbeforefirst.apx}
\begin{description}
\item[Short] 
 Shows how to start continuous filters etc before the first trial is presented
\item[Description] 
 The noise is playing continuously over trials (dataloop)  - the noise starts 5s before the first trial (presenting: wd1.wav: 'een') begins
\item[How] 
 procedure: time\_before\_first\_trial: in seconds - filters: dataloop - continuous: true
\end{description}

\subsubsection{procedure/cst-multistimulus.apx}
\begin{description}
\item[Short] 
 multiple stimuli per trial - one should be chosen randomly - 2 presentations (2trials =\textgreater{} 2x2 = 4)
\item[Description] 
 The experiment has 2 trials (trial1: button 1-2-3 - trial2: button a-b-c) =\textgreater{} In trial1/trial2=\textgreater{} correct answer is always: button1,buttonb.
\item[How] 
 constantProcedure - 2 trials with !each! 3 different stimuli but with the !same! buttons
\end{description}

\subsubsection{procedure/extrasimple.apx}
\begin{description}
\item[Short] 
 Click on the corresponding button - Stop after 2 presentations (2trials =\textgreater{} 2x2 = 4)
\item[Description] 
 The experiment has 2 trials with different stimuli (trial1: house - trial2: mouse) =\textgreater{} In trial1/trial2 =\textgreater{} correct answer is: house/mouse (stimulus is heard in the left ear)
\item[How] 
 constantProcedure - 2 trials with different stimuli corresponding with the buttons
\end{description}

\subsubsection{procedure/fixedparameternotfound.apx}
\begin{description}
\item[Short] 
 Errormessage - fixed parameter with id="test" not defined
\item[Description] 
 example of an error message about an undefined fixed parameter of a stimulus
\item[How] 
 stimuli - fixed parameters
\end{description}

\subsubsection{procedure/heartrainprocedure.apx}
\begin{description}
\item[Short] 
 Regression test for heartrainprocedure.js
\item[Description] 
 example of an error message about an undefined fixed parameter of a stimulus
\item[How] 
 stimuli - fixed parameters
\end{description}

\subsubsection{procedure/heartrainprocedure\_short.apx}
\begin{description}
\item[Short] 
\item[Description] 
\item[How] 
\end{description}

\subsubsection{procedure/idn1.apx}
\begin{description}
\item[Short] 
 Matching of stimuli and buttons - Different trials (+ answers) - 1 presentations (6 trials)
\item[Description] 
 auditive stimulus 1 2 3 4 5 6 - input: buttons 1 2 3 4 5 6 =\textgreater{} match the stimulus with the button
\item[How] 
 6 trials with each trial a different correct answer - 6 stimuli and 6 buttons - order: sequential
\end{description}

\subsubsection{procedure/idn1-mono.apx}
\begin{description}
\item[Short] 
 Matching of stimulus and button - 5 presentations (1 trial)
\item[Description] 
 auditive stimulus in right ear '1' - input: buttons 1 2 3 4 5 6 =\textgreater{} match the stimulus with the button (stimulus1 =\textgreater{} button1)
\item[How] 
 1 trial - 6 stimuli and 6 buttons - 1 correct answer
\end{description}

\subsubsection{procedure/idn1-skip.apx}
\begin{description}
\item[Short] 
 skip = 2: Number of trials that will be presented before the actual presentations start.
\item[Description] 
 skip=2 and presentations=2 =\textgreater{} first 2 trials and then 2*6trials = 12 trials. If the order is sequential, the skipped trials will be the first skip trials from the trial list, repeated if necessary.
\item[How] 
 2 presentations - 6 trials - skip = 2 (6 stimuli - 6 buttons)
\end{description}

\subsubsection{procedure/idn1-waitforstart.apx}
\begin{description}
\item[Short] 
 Demonstrate use of wait for start
\item[Description] 
 The next trial is only presented after selecting Start from the Experiment menu or pressing F5.
\item[How] 
 \textless{}waitforstart\textgreater{} function in \textless{}general\textgreater{}
\end{description}

\subsubsection{procedure/idn2.apx}
\begin{description}
\item[Short] 
 Matching of stimuli and buttons - Different trials (+ answers) - 1 presentations (6 trials)
\item[Description] 
 auditive stimulus 1 2 3 4 5 6 - input: buttons 1 2 3 4 5 6 =\textgreater{} match the stimulus with the button
\item[How] 
 6 trials with each trial a different correct answer - 6 stimuli and 6 buttons - order: random
\end{description}

\subsubsection{procedure/idn2-skip.apx}
\begin{description}
\item[Short] 
 skip = 3: Number of trials that will be presented before the actual presentations start.
\item[Description] 
 skip=3 and presentations=2 =\textgreater{} first 3 trials and then 2*6trials = 12 trials. If the order is random, the skipped trials will be picked from the trial list without replacement, repeating this procedure if necessary.
\item[How] 
 2 presentations - 6 trials - skip = 3 (6 stimuli - 6 buttons)
\end{description}

\subsubsection{procedure/input-during-stimulus.apx}
\begin{description}
\item[Short] 
 Input during stimulus is allowed
\item[Description] 
 skip=3 and presentations=2 =\textgreater{} first 3 trials and then 2*6trials = 12 trials. If the order is random, the skipped trials will be picked from the trial list without replacement, repeating this procedure if necessary.
\item[How] 
 Input during stimulus: true - trials with buttons (input) - stimuli (output)
\end{description}

\subsubsection{procedure/kaernbach.apx}
\begin{description}
\item[Short] 
 1up-1down Kaernbach procedure: Frequency of ups-downs is the same - Up stepsize is 1, down stepsize is 7 - Experiment stops after 6 reversals
\item[Description] 
 When you click on the button 'correct or wrong', you hear a stimulus 'wd1.wav (een/one)' in both ears. When you click again, the gain of the stimulus decreases/increases according to the button (correct/false)
\item[How] 
 adaptiveProcedure - stop\_after\_type: reversals - nUp/nDown - adapt\_parameter (gain: gain of stimulus) -
\end{description}

\subsubsection{procedure/multiparameters-fixed.apx}
\begin{description}
\item[Short] 
 Adaptation of multiple parameters (order(fixed) \& gain(adapt))
\item[Description] 
 Two buttons: louder - quieter =\textgreater{} louder: increasing of order and gain with stepsize=2, quieter: decreasing of order and gain with stepsize=2 (see output box)
\item[How] 
 Adapt\_parameter =\textgreater{} order \& gain - stepsize for gain(filter) \& order(stimuli) = 2
\end{description}

\subsubsection{procedure/multiparameters-invalid.apx}
\begin{description}
\item[Short] 
 Error message - parameter you want to be adapted is a fixed parameter (2nd adapt\_parameter: order: is a fixed parameter)
\item[Description] 
 error message: only first adaptive parameter can be a fixed parameter
\item[How] 
 Adapt\_parameter =\textgreater{} gain(adapt) \& order(fixed)
\end{description}

\subsubsection{procedure/multiparameters-variable.apx}
\begin{description}
\item[Short] 
 Adaptation of multiple parameters: GainL \& GainR
\item[Description] 
 2 buttons: louder - quieter - when you click on louder/quieter the gain in right and left channel is increased/decreased with 2 dB (stepsize) - see output box
\item[How] 
 Adapt\_parameter =\textgreater{} gainL(adapt) \& gainR(adapt)
\end{description}

\subsubsection{procedure/multiprocedure-aid.apx}
\begin{description}
\item[Short] 
 Multiprocedure - 2 adaptiveprocedures with each their own parameters, procedure and trials
\item[Description] 
 2 procedures =\textgreater{} procedure1/procedure2: 1trial - button\_correct/button\_wrong - When you click correct, the experiment in the next trial of the same procedure becomes more difficult (you hear a higher number =\textgreater{} larger value of snr = stimulus (twee, drie, vier, vijf: stepsize:1)
\item[How] 
 procedure =\textgreater{} multiProcedure - larger\_is\_easier =\textgreater{} snr(not in dB): larger value: easier (so if you click on the wrong\_button =\textgreater{} the experiment becomes easier)
\end{description}

\subsubsection{procedure/multiprocedure.apx}
\begin{description}
\item[Short] 
 Multiprocedure - 2 constantprocedures with each their own parameters, procedure and trials
\item[Description] 
 2 procedures =\textgreater{} procedure1/procedure2: 3 trials - button1/button4 - button2/button5 - button3/button6 but different stimuli in each trial (Match the stimulus with the button)
\item[How] 
 procedure =\textgreater{} multiProcedure
\end{description}

\subsubsection{procedure/multiprocedure-choices.apx}
\begin{description}
\item[Short] 
 Multiprocedure - 2 constantprocedures with each their own parameters, procedure and trials - 2 different choices of intervals
\item[Description] 
 2 procedures =\textgreater{} procedure1/procedure2: 1trial - 3intervals - Match the stimulus (place of stimulus within noise) with the correct button
\item[How] 
 procedure =\textgreater{} multiProcedure - intervals/choices - 3 choices - standardstimulus:noise
\end{description}

\subsubsection{procedure/multiprocedure-choices-corrector.apx}
\begin{description}
\item[Short] 
 Multiprocedure - 2 constantprocedures with each their own parameters, procedure and trials, \#presentations
\item[Description] 
 2 procedures =\textgreater{} procedure1: 3presentations/1trial =\textgreater{} 3x - 3intervals: Match the stimulus (place of stimulus within noise) with the correct button
\item[How] 
 procedure =\textgreater{} multiProcedure - intervals/choices - standardstimulus:noise
\end{description}

\subsubsection{procedure/multiprocedure-constant-train.apx}
\begin{description}
\item[Short] 
 Multiprocedure - 2procedures (1constant - 1training) with each their own parameters, procedure and trials
\item[Description] 
 2 procedures =\textgreater{} constantprocedure1: 4presentations/1trial =\textgreater{} 4x - 3intervals: Match the stimulus (place of stimulus within noise) with the correct button
\item[How] 
 procedure =\textgreater{} multiProcedure - intervals/choices - standardstimulus:noise - constantProcedure - TrainingProcedure
\end{description}

\subsubsection{procedure/multiprocedure-idn.apx}
\begin{description}
\item[Short] 
 Multiprocedure - 2constantprocedures with each their own parameters, procedure and trials
\item[Description] 
 2 procedures =\textgreater{} constantprocedure1: order: onebyone - 4presentations/1trial =\textgreater{} 4x - 3intervals: Match the stimulus (place of stimulus within noise) with the correct button
\item[How] 
 procedure =\textgreater{} multiProcedure - intervals/choices - standardstimulus:noise - constantProcedure
\end{description}

\subsubsection{procedure/multiprocedure-mixed.apx}
\begin{description}
\item[Short] 
 Multiprocedure - 2constantprocedures with the SAME procedure, parameters and trials
\item[Description] 
 2 procedures =\textgreater{} constantprocedure1: 1presentations/3trials =\textgreater{} 3x - 3buttons: 1-2-3 Match the stimulus (place of stimulus within noise) with the correct button
\item[How] 
 procedure =\textgreater{} multiProcedure - constantProcedure
\end{description}

\subsubsection{procedure/multiprocedure-onebyone.apx}
\begin{description}
\item[Short] 
 Multiprocedure-order! - 3constantprocedures with the SAME procedure, parameters and trials (different stimuli)
\item[Description] 
 2 procedures =\textgreater{} constantprocedure1: 2presentations/1trial =\textgreater{} 2x - 2buttons: name of procedure \& number: click on number (=1) if you have heard the stimulus (corresponding with the number)
\item[How] 
 procedure =\textgreater{} multiProcedure: order: ONEBYONE - constantProcedure
\end{description}

\subsubsection{procedure/multiprocedure-random.apx}
\begin{description}
\item[Short] 
 Multiprocedure-order! - 3constantprocedures with the SAME procedure, parameters and trials (different stimuli)
\item[Description] 
 2 procedures =\textgreater{} constantprocedure1: 2presentations/1trial =\textgreater{} 2x - 2buttons: name of procedure \& number: click on number (=1) if you have heard the stimulus (corresponding with the number)
\item[How] 
 procedure =\textgreater{} multiProcedure: order: RANDOM - constantProcedure
\end{description}

\subsubsection{procedure/multiprocedure-train-train.apx}
\begin{description}
\item[Short] 
 Multiprocedure-order - 2trainingProcedures with DIFFERENT procedure, parameters and trials (different stimuli)
\item[Description] 
 2 procedures =\textgreater{} trainingProcedure1: 3trials =\textgreater{} 3x???? - 3buttons: click on a button and hear the corresponding stimulus
\item[How] 
 procedure =\textgreater{} multiProcedure: order: ONEBYONE- trainingProcedure
\end{description}

\subsubsection{procedure/multiscreen-idn1.apx}
\begin{description}
\item[Short] 
 Multiscreen - 1 procedure with for the first trials: screen 1 and the last trials: screen 2 =\textgreater{} Same outcome as sequential Multiprocedure
\item[Description] 
 6 trials - 1 presentation: 3 buttons(screen1): 1-2-3 and 3 buttons(screen2): 4-5-6 =\textgreater{} Match the stimulus with the corresponding button
\item[How] 
 procedure =\textgreater{} 1procedure but several screens: multiscreen- constantProcedure
\end{description}

\subsubsection{procedure/multistandard-unique.apx}
\begin{description}
\item[Short] 
 Multiple standards - Unique standard
\item[Description] 
 2 trials: trial1 = match one of the button with the noise-stimulus (standards/reference-signals = numbers)
\item[How] 
 2 trials - constantProcedure
\end{description}

\subsubsection{procedure/noanswer.apx}
\begin{description}
\item[Short] 
 Show warning if no answer is defined for a stimulus
\item[Description] 
 Error message =\textgreater{} Cannot show feedback because: no screen was found for (button turns red)
\item[How] 
 procedure - trial1 - no answer (see comment)
\end{description}

\subsubsection{procedure/nostandardfound.apx}
\begin{description}
\item[Short] 
 Show warning if no answer is defined for a stimulus
\item[Description] 
 Error message =\textgreater{} Cannot show feedback because: no screen was found for (button turns red)
\item[How] 
 procedure - trial1 - no answer (see comment)
\end{description}

\subsubsection{procedure/open-set-constant.apx}
\begin{description}
\item[Short] 
 Open set experiment - Constant stimuli
\item[Description] 
 6 trials (1 presentation) - You hear a stimulus (1-2-3-4-5-6) in both ears - Typ the answer/stimulus (een, twee, drie, ...) in the textbox below 'answer'.
\item[How] 
 trial - answer\textgreater{}ANSWER THAT THE SUBJECT HAS TO TYPE\textless{} - corrector xsi:type="apex:isequal"
\end{description}

\subsubsection{procedure/pause\_between\_stimuli.apx}
\begin{description}
\item[Short] 
  a pause of 500ms should be introduced between the stimulus and the standards
\item[Description] 
 1 trials (10 presentations =\textgreater{} 10x) - You hear a stimulus (1-2-3-4-5-6) in both ears - Typ the answer/stimulus (een, twee, drie, ...) in the textbox below 'answer'.
\item[How] 
 pause between stimuli (in seconds)
\end{description}

\subsubsection{procedure/pluginprocedure.apx}
\begin{description}
\item[Short] 
 Pluginprocedure =\textgreater{} script: testprocedure
\item[Description] 
 When you click on a button with a certain number, the next time you have to click on the number you just have heard - parameter: stepsize \& startvalue
\item[How] 
 pluginProcedure with a certain script
\end{description}

\subsubsection{procedure/pluginprocedure-matrixtest.apx}
\begin{description}
\item[Short] 
\item[Description] 
\item[How] 
\end{description}

\subsubsection{procedure/randomchannel.apx}
\begin{description}
\item[Short] 
 Randomgenerator test
\item[Description] 
 the stimulus "een" \& "twee" are played simultaneously but randomly in the left or right channel according the value returnd by random1
\item[How] 
 trainingProcedure - randomgenerator - connections - stimuli:datablocks: simultaneous
\end{description}

\subsubsection{procedure/repeatuntillcorrect-endaftertrials.apx}
\begin{description}
\item[Short] 
 The first selected trial (of the first presentation) shoud be repeatedly presented (stop after 5 presentations, several trials)
\item[Description] 
 5 presentations x 5 trials = 25x - When you click correct =\textgreater{} the next trial begins / When you click wrong, the first trial of the first presentation is repeated
\item[How] 
 repeat\_first\_until\_correct: true
\end{description}

\subsubsection{procedure/repeatuntillcorrect-multitrial.apx}
\begin{description}
\item[Short] 
 The first selected trial (of the first presentation) shoud be repeatedly presented (stop after 5 reversals, multiple trials)
\item[Description] 
 5 trials, stop after 5 reversals - When you click correct =\textgreater{} the next trial begins / When you click wrong, the first trial of the first presentation is repeated
\item[How] 
 adaptiveprocedure - REPEAT-FIRST-UNTIL-CORRECT: TRUE - adapt\_parameter
\end{description}

\subsubsection{procedure/repeatuntillcorrect-singletrial.apx}
\begin{description}
\item[Short] 
 The first selected trial shoud be repeatedly presented: DOESN'T WORK! =\textgreater{} single trial + adapt\_parameter
\item[Description] 
 1 trial, stop after 5 reversals - When you click correct =\textgreater{} the next trial begins / When you click wrong, the first trial begins
\item[How] 
 adaptiveprocedure - repeat\_first\_until\_correct: true - ADAPT\_PARAMETER
\end{description}

\subsubsection{procedure/selectrandom.apx}
\begin{description}
\item[Short] 
 Test of random stimulus output when more than one stimulus defined in the experiment
\item[Description] 
 Random selection of stimuli in a trial if there are more than one (=\textgreater{} sequential presentation of stimuli =\textgreater{} more trials) - When you click on 1, 'een', 'twee', 'drie' is presented randomly in both ears
\item[How] 
 constantprocedure - more than 1 stimulus in 1 trial =\textgreater{} random selection of the stimulus
\end{description}

\subsubsection{procedure/simple.apx}
\begin{description}
\item[Short] 
 Randomgenerator test
\item[Description] 
 the stimulus "house"/"mouse" are played in the left channel =\textgreater{} click the right button (house/mouse)
\item[How] 
 constantProcedure - stimuli - datablocks
\end{description}

\subsubsection{procedure/soundquality.apx}
\begin{description}
\item[Short] 
 Interval - select element ok, number 2
\item[Description] 

\item[How] 

\end{description}

\subsubsection{procedure/time-between-trials.apx}
\begin{description}
\item[Short] 
 Demonstrates the use of feedback to enforce a pauze between trials
\item[Description] 
 If you click on a button, you hear a stimulus (1-2-3-4-5-6) in both ears =\textgreater{} click on the button corresponding with the stimulus
\item[How] 
 feedback length (in ms)
\end{description}

\subsubsection{procedure/train1.apx}
\begin{description}
\item[Short] 
 Trainingprocedure: demonstrate the presentation of a stimulus with a number according to last input
\item[Description] 
 when you click on 1/2/3/4/5/6 =\textgreater{} you hear een/twee/drie/vier/vijf/zes
\item[How] 
 trainingProcedure with 6 trials and buttons
\end{description}

\subsubsection{procedure/train2.apx}
\begin{description}
\item[Short] 
 Show the presentation of a stimulus with a random number out of the possible choices according to last input
\item[Description] 
 3 buttons: 1-2  3-4  5-6 =\textgreater{} when you click on 5-6, you hear either 5 or 6
\item[How] 
 trainingProcedure - multiple trials =\textgreater{} button5 is the answer of the trial corresponding with stimulus5 \& stimulus6
\end{description}

\subsubsection{procedure/trial-nostimulus.apx}
\begin{description}
\item[Short] 
 Make a trial whithout a stimulus
\item[Description] 
 when you click on stimulus =\textgreater{} apex shuts down
\item[How] 
 no stimuli - datablock
\end{description}

\subsubsection{procedure/uniform-double.apx}
\begin{description}
\item[Short] 
 Random generator test - gain of stimulus is randomly chosen
\item[Description] 
 You hear 'een' in your left ear - the gain of this stimulus changes randomly, each time you click the button '1'
\item[How] 
 random generator - type: DOUBLE (fractional number, e.g. 0.1) - parameter =\textgreater{} see filter
\end{description}

\subsubsection{procedure/uniform-int.apx}
\begin{description}
\item[Short] 
 Random generator test - gain of stimulus is randomly chosen
\item[Description] 
 You hear 'een' in your left ear - the gain of this stimulus changes randomly, each time you click the button '1'
\item[How] 
 random generator - type: WHOLE(fractional number, e.g. -5) - parameter =\textgreater{} see filter
\end{description}

\subsection{randomgenerator}
\subsubsection{randomgenerator/multi-interval.apx}
\begin{description}
\item[Short] 
 Randomgenerator: random gain of each datablock of a multi-interval stimulus
\item[Description] 
 The stimulus "één één één" is heard each time by pressing on the "1" button, and the gain of each "één" has a random gain
\item[How] 
 The id of the gain is set to be the parameter of the randomgenerator, three intervals are defined in procedure
\end{description}

\subsubsection{randomgenerator/randomchannel.apx}
\begin{description}
\item[Short] 
 Randomgenerator: random channel for stimulus
\item[Description] 
 The stimulus "één" is played on randomly the left or right channel
\item[How] 
 The value returnd by random1 decides whether it is channel 0 or channel 1, see \textless{}randomgenerators\textgreater{}
\end{description}

\subsubsection{randomgenerator/uniform-double.apx}
\begin{description}
\item[Short] 
 Randomgenerator: random gain of stimulus presented in 1 channel, possibility to use non-integer values (double type)
\item[Description] 
 The stimulus "één" is heard in the right channel each time by pressing on the "1" button, and the gain of "één" has a random gain
\item[How] 
 See \textless{}randomgenerators\textgreater{}, the id of the gain is set to be the parameter of the randomgenerator, minimum value contains a decimal value
\end{description}

\subsubsection{randomgenerator/uniform-int.apx}
\begin{description}
\item[Short] 
 Randomgenerator: random gain of stimulus presented in 1 channel, only possible to use integer values
\item[Description] 
 The stimulus "één" is heard in the left channel each time by pressing on the "1" button, and the gain of "één" has a random gain
\item[How] 
 See \textless{}randomgenerators\textgreater{}, the id of the gain is set to be the parameter of the randomgenerator, minimum and maximum values are integers
\end{description}

\subsubsection{randomgenerator/uniform-int-invalid.apx}
\begin{description}
\item[Short] 
 Randomgenerator should have integer values, similar to uniform-int.apx
\item[Description] 
 Warning (no error) appears in messages window because random generator limits are not integer
\item[How] 
 Randomgenerator max value is 2.5
\end{description}

\subsubsection{randomgenerator/uniform-smallint.apx}
\begin{description}
\item[Short] 
 Randomgenerator: random gain of stimulus presented in 1 channel, only possible to use integer values, only small range is used
\item[Description] 
 The stimulus "één" is heard in the left channel each time by pressing on the "1" button, and the gain of "één" has a random gain (small range in this case)
\item[How] 
 See \textless{}randomgenerators\textgreater{}, the id of the gain is set to be the parameter of the randomgenerator, minimum and maximum values are integers
\end{description}

\subsection{results}
\subsubsection{results/results-adaptive-saveprocessedresults.apx}
\begin{description}
\item[Short] 
 Demonstrate the use of result parameters
\item[Description] 
 The parameters under 'results' will be written to the results file.
\item[How] 
 Use of rtresults-test-procedureparameter.html
\end{description}

\subsubsection{results/results-confusionmatrix.apx}
\begin{description}
\item[Short] 
 Demonstrate the use of a confusionmatrix
\item[Description] 
 Results are converted into a confusionmatrix
\item[How] 
 Default results file for a confusionmatrix
\end{description}

\subsubsection{results/results-procedureparameter-after.apx}
\begin{description}
\item[Short] 
 Demonstrate the use of result parameters
\item[Description] 
 Results are shown after the experiment is done.
\item[How] 
 Results: show results after experiment
\end{description}

\subsubsection{results/results-procedureparameter-during.apx}
\begin{description}
\item[Short] 
 Demonstrate the use of real-time results
\item[Description] 
 Real-time results are shown during the experiment in a separate window
\item[How] 
 Results: show results during experiment
\end{description}

\subsubsection{results/results-psignifit.apx}
\begin{description}
\item[Short] 
 Demonstrate the use of 'psignifit' results
\item[Description] 
 APEX contains an implementation of the psignifit library, which can fit psychometric functions to data
\item[How] 
 Psignifit is called from the resultsviewer and the SRT and slope are shown numerically
\end{description}

\subsubsection{results/results-resultparameters.apx}
\begin{description}
\item[Short] 
 Demonstrate the use of resultparameters
\item[Description] 
 Resultparameters specified in \textless{}results\textgreater{}/\textless{}resultparameters\textgreater{} will be passed on to the resultviewer
\item[How] 
 These parameters will be made available in a hash params. 
\end{description}

\subsubsection{results/results-resultparams-localization-polarplot.apx}
\begin{description}
\item[Short] 

\item[Description] 

\item[How] 

\end{description}

\subsubsection{results/results-saveprocessedresults.apx}
\begin{description}
\item[Short] 
 Demonstrate the use of the saveprocessedresults tag
\item[Description] 
 Results in CSV format will be appended to the results file, for easy importing in other software
\item[How] 
 \textless{}saveprocessedresults\textgreater{} is set to true
\end{description}

\subsubsection{results/results-subject.apx}
\begin{description}
\item[Short] 
 Demonstrate how you can add the subjects name to the results file
\item[Description] 
 When the subject types his/her name in the interactive, this will automatically appear in the results file and be appended to the results filename
\item[How] 
 subject function in 'results', 'interactive' entry is added
\end{description}

\subsection{screen}
\subsubsection{screen/button-shortcuts.apx}
\begin{description}
\item[Short] 
 shows how to use shortcuts to answer
\item[Description] 
 The screen shows different buttons: you can either click on them, or use a predefined shortcut button on the keyboard
\item[How] 
 shortcuts are implemented for the buttons
\end{description}

\subsubsection{screen/currentfeedback.apx}
\begin{description}
\item[Short] 
 highlighting of the played stimulus
\item[Description] 
 the currently playing stimulus/button is highlighted
\item[How] 
 \textless{}showcurrent\textgreater{}true
\end{description}

\subsubsection{screen/feedback-answer.apx}
\begin{description}
\item[Short] 
 feedback with picture and highlighting  of the CLICKED element
\item[Description] 
 When clicking a button, a feedback picture (thumb up or down) is shown in the right panel and the CLICKED button is highlighted.
\item[How] 
 \textless{}feedback\_on\textgreater{} clicked
\end{description}

\subsubsection{screen/feedback-both.apx}
\begin{description}
\item[Short] 
 highlighted stimulus + feedback with picture and highlighting
\item[Description] 
 The currently playing stimulus/button is highlighted. When clicking a button, a feedback picture (thumb up or down) is shown in the right panel and the correct button is highlighted.
\item[How] 
 \textless{}showcurrent\textgreater{} true, \textless{}feedback\textgreater{} true
\end{description}

\subsubsection{screen/feedback-ledfeedback.apx}
\begin{description}
\item[Short] 

\item[Description] 

\item[How] 

\end{description}

\subsubsection{screen/feedback-ledfeedback-localization.apx}
\begin{description}
\item[Short] 

\item[Description] 

\item[How] 

\end{description}

\subsubsection{screen/feedback-ledfeedback-showcurrent.apx}
\begin{description}
\item[Short] 

\item[Description] 

\item[How] 

\end{description}

\subsubsection{screen/feedback-multistimboth.apx}
\begin{description}
\item[Short] 
 highlighted stimulus + feedback with picture and highlighting for multiple stimuli
\item[Description] 
 The currently playing stimulus/button is highlighted. When clicking a button, a feedback picture (thumb up or down) is shown in the right panel and the correct button is highlighted.
\item[How] 
 \textless{}showcurrent\textgreater{} true, \textless{}feedback\textgreater{} true
\end{description}

\subsubsection{screen/feedback-multistimnormal.apx}
\begin{description}
\item[Short] 
 feedback with picture and highlighting for multiple stimuli
\item[Description] 
 When clicking a button, a feedback picture (thumb up or down) is shown in the right panel and the correct button is highlighted.
\item[How] 
 \textless{}feedback\textgreater{} true
\end{description}

\subsubsection{screen/feedback-multistimshowcurrent.apx}
\begin{description}
\item[Short] 
 highlighted stimulus for multiple stimuli
\item[Description] 
 the currently playing stimulus/button is highlighted
\item[How] 
 \textless{}showcurrent\textgreater{} true
\end{description}

\subsubsection{screen/feedback-nohighlight.apx}
\begin{description}
\item[Short] 
 feedback without highlighting
\item[Description] 
 When clicking a button, a feedback picture (thumb up or down) is shown in the right panel and NO elements are highlighted.
\item[How] 
 \textless{}feedback\_on\textgreater{}: none
\end{description}

\subsubsection{screen/feedback-normal.apx}
\begin{description}
\item[Short] 
 feedback with picture and highlighting
\item[Description] 
 When clicking a button, a feedback picture (thumb up or down) is shown in the right panel and the correct button is highlighted.
\item[How] 
 \textless{}feedback\textgreater{} true
\end{description}

\subsubsection{screen/feedback-onlywait.apx}
\begin{description}
\item[Short] 
 no visual feedback
\item[Description] 
 No feedback is given after clicking the correct/wrong button (no highlighting/picture). Subject has to wait between trials.
\item[How] 
 \textless{}feedback length= ...\textgreater{} false
\end{description}

\subsubsection{screen/feedback-plugin.apx}
\begin{description}
\item[Short] 
 Feedback using a plugin
\item[Description] 
 Own feedback can be tested if dummyfeedbackplugin is changed
\item[How] 
 dummyfeedbackplugin
\end{description}

\subsubsection{screen/feedback-showcurrent.apx}
\begin{description}
\item[Short] 
 highlighted stimulus
\item[Description] 
 the currently playing stimulus/button is highlighted
\item[How] 
 \textless{}showcurrent\textgreater{} true
\end{description}

\subsubsection{screen/fgcolor.apx}
\begin{description}
\item[Short] 
 shows how to change the font color of a button/label
\item[Description] 
 A label and button with a different font color are shown on the screen
\item[How] 
 \textless{}fgcolor\textgreater{}
\end{description}

\subsubsection{screen/flash-disabled.apx}
\begin{description}
\item[Short] 
 Shows full use of flash movies, but without the actual childmode. Shows how to disable a button.
\item[Description] 
 the middle egg should not be clickable (disabled)
\item[How] 
 \textless{}disabled\textgreater{}true
\end{description}

\subsubsection{screen/flash-flashfeedback.apx}
\begin{description}
\item[Short] 
 Shows full use of flash movies, but without the actual childmode
\item[Description] 
 Corresponding flash movies are shown during presentation of the stimuli and after clicking the correct/wrong button.
\item[How] 
 introduce flash elements: \textless{}flash\textgreater{}, \textless{}uri\textgreater{}, \textless{}feedback\textgreater{}
\end{description}

\subsubsection{screen/flash-normalfeedback.apx}
\begin{description}
\item[Short] 
 Pictures instead of buttons
\item[Description] 
 Three eggs instead of buttons are shown.
\item[How] 
 flashelements for screen
\end{description}

\subsubsection{screen/fullscreen.apx}
\begin{description}
\item[Short] 
 shows how to conduct an experiment in full screen modus
\item[Description] 
 A typical experiment is shown, but the Windows titlebar/taskbar/... is hidden
\item[How] 
 \textless{}fullscreen\textgreater{}true\textless{}fullscreen\textgreater{}
\end{description}

\subsubsection{screen/general-itiscreen-annoyingtext.apx}
\begin{description}
\item[Short] 
 shows how to use an intertrial screen
\item[Description] 
 A screen is shown between two trials.
\item[How] 
 \textless{}screens\textgreater{} \textless{}general\textgreater{} \textless{}intertrialscreen\textgreater{}iti
\end{description}

\subsubsection{screen/general-itiscreen-picture.apx}
\begin{description}
\item[Short] 
 shows how to use an intertrial picture
\item[Description] 
 A picture is shown between two trials.
\item[How] 
 \textless{}screens\textgreater{} \textless{}general\textgreater{} \textless{}intertrialscreen\textgreater{}iti
\end{description}

\subsubsection{screen/layout-arc.apx}
\begin{description}
\item[Short] 
 example of different arc layouts for buttons
\item[Description] 
 Buttons are placed in several semi-circles on the screen
\item[How] 
 \textless{}arcLayout\textgreater{} upper, lower, left, right, full
\end{description}

\subsubsection{screen/layout-arc-invalidwidth.apx}
\begin{description}
\item[Short] 
 Shows problems with arc layout for buttons
\item[Description] 
 Error when loading the experiment: something went wrong with the placing of the buttons.
\item[How] 
 \textless{}arcLayout\textgreater{} The number of x-values can't exceed the width of the arc. In this case, the width has to be changed to 9 (instead of 2)
\end{description}

\subsubsection{screen/layout-arcs.apx}
\begin{description}
\item[Short] 
 example of different arc layouts for buttons
\item[Description] 
 Buttons are placed in several semi-circles on the screen
\item[How] 
 \textless{}arcLayout\textgreater{} upper, lower, left, right, full
\end{description}

\subsubsection{screen/layout-arc-single.apx}
\begin{description}
\item[Short] 
 example of simple archwise organisation of buttons
\item[Description] 
 Buttons are placed archwise/in a semi-circle on the screen.
\item[How] 
 \textless{}arcLayout\textgreater{} upper, lower, left or right
\end{description}

\subsubsection{screen/layout-grid.apx}
\begin{description}
\item[Short] 
 example of gridlayout
\item[Description] 
 buttons are placed in a grid
\item[How] 
 \textless{}gridLayout\textgreater{}
\end{description}

\subsubsection{screen/layout-grid-nested.apx}
\begin{description}
\item[Short] 
 example of nested gridlayout
\item[Description] 
 buttons are placed in nested grids
\item[How] 
 \textless{}gridLayout\textgreater{} in \textless{}gridLayout\textgreater{}
\end{description}

\subsubsection{screen/layout-grid-rowcol.apx}
\begin{description}
\item[Short] 
 shows the use of the row and col attributes instead of x and y
\item[Description] 
 first screen you see is made with col \& row, second screen with x \& y
\item[How] 
 \textless{}gridLayout\textgreater{} replacing "x" \& "y" by "row" \& "col". col = column in the grid, is the same as x. row = row in the grid, is the same as y
\end{description}

\subsubsection{screen/layout-grid-rowcol-invalid.apx}
\begin{description}
\item[Short] 
 shows wrong usage of the row and col attributes
\item[Description] 
 first screen you see is made with col \& row, second screen with x \& y
\item[How] 
 \textless{}gridLayout\textgreater{} col should not be used together with x! col is the same as x, refers to column in the grid
\end{description}

\subsubsection{screen/layout-grid-stretchfactors.apx}
\begin{description}
\item[Short] 
 shows how to change the size of the buttons
\item[Description] 
 Buttons showed on the screen have different sizes/stretchfactors.
\item[How] 
 Stretch factor for the columns/rows: a list of integers separated by comma's. The width of the columns/rows will be proportional to the numbers.
\end{description}

\subsubsection{screen/layout-grid-stretchfactors-invalid.apx}
\begin{description}
\item[Short] 
 Shows problems with changing the size of the buttons
\item[Description] 
 Error when loading the experiment: something went wrong when changing the size of the buttons.
\item[How] 
 The number of the column stretch factors has to be equal to its width.
\end{description}

\subsubsection{screen/layout-hlayout.apx}
\begin{description}
\item[Short] 
 shows horizontal layout
\item[Description] 
 buttons are placed next to each other, in a horizontal layout
\item[How] 
 \textless{}hLayout\textgreater{}
\end{description}

\subsubsection{screen/layout-mixed.apx}
\begin{description}
\item[Short] 
 example of mixed layouts
\item[Description] 
 buttons are placed in semi-circles and grids on the screen
\item[How] 
 \textless{}arcLayout\textgreater{}, \textless{}gridLayout\textgreater{}
\end{description}

\subsubsection{screen/nomenu.apx}
\begin{description}
\item[Short] 
 shows how to hide the menu bar
\item[Description] 
 A typical experiment is shown, but the menu bar on top of the screen is hidden.
\item[How] 
 \textless{}showmenu\textgreater{}false\textless{}showmenu\textgreater{}
\end{description}

\subsubsection{screen/panel-feedbackpictures.apx}
\begin{description}
\item[Short] 
 shows how to implement a feedback picture, other than the traditional thumb
\item[Description] 
 A green square is shown as feedback when clicking the correct button, a red square is shown when you give a wrong answer
\item[How] 
 \textless{}feedback\_picture\_positive\textgreater{} \& \textless{}feedback\_picture\_negative\textgreater{}
\end{description}

\subsubsection{screen/panel-feedbackpictures-filenotfound.apx}
\begin{description}
\item[Short] 
 shows error message for feedback picture
\item[Description] 
 Experiment doesn't load, because the feedback pictures are not found.
\item[How] 
 Wrong naming of feedback pictures / pictures don't exist
\end{description}

\subsubsection{screen/panel-nopanel.apx}
\begin{description}
\item[Short] 
 shows how to hide the panel
\item[Description] 
 A typical experiment is shown, but the panel on the right side is hidden. Start the experiment by pressing F5.
\item[How] 
 \textless{}showpanel\textgreater{}false\textless{}showpanel\textgreater{}
\end{description}

\subsubsection{screen/panel-showrepeatbutton.apx}
\begin{description}
\item[Short] 
 shows the use of a repeatbutton
\item[Description] 
 A repeatbutton is shown on the screen, just above the progressbar. When clicking the button, the last stimulus is repeated.
\item[How] 
 \textless{}repeatbutton\textgreater{}true\textless{}repeatbutton\textgreater{}
\end{description}

\subsubsection{screen/panel-showstatuspicture.apx}
\begin{description}
\item[Short] 
 shows the use of a statuspicture
\item[Description] 
 A statuspicture (dot) is shown, just above the progress bar. The dot changes color: red when presenting the stimulus, green when waiting for an answer
\item[How] 
 \textless{}statuspicture\textgreater{}true\textless{}statuspicture\textgreater{}
\end{description}

\subsubsection{screen/panel-showstopandrepeatbutton.apx}
\begin{description}
\item[Short] 
 shows how to use a stopbutton \& repeatbutton
\item[Description] 
 A red stopbutton and a repeatbutton are displayed on the screen, just above the progressbar. When clicking the stopbutton, all output is immediately stopped \& apex is shut down. When clicking the repeatbutton, the last stimulus is repeated.
\item[How] 
 \textless{}stopbutton\textgreater{} \& \textless{}repeatbutton\textgreater{}
\end{description}

\subsubsection{screen/panel-showstopbutton.apx}
\begin{description}
\item[Short] 
 shows how to use the stopbutton
\item[Description] 
 A red stopbutton is displayed on the screen, just above the progressbar. When clicking this button, all output is immediately stopped \& apex is shut down
\item[How] 
 \textless{}stopbutton\textgreater{}true\textgreater{}stopbutton\textgreater{}
\end{description}

\subsubsection{screen/prefix-apexconfig.apx}
\begin{description}
\item[Short] 
 shows how to use an 'apexconfig' as a prefix for a filename
\item[Description] 
  idem
\item[How] 
 \textless{}uri\_prefix source="apexconfig"\textgreater{}pictures\textless{}/uri\_prefix\textgreater{}
\end{description}

\subsubsection{screen/prefix-inline.apx}
\begin{description}
\item[Short] 
 shows how to use an 'inline' as a prefix for a filename
\item[Description] 
 idem
\item[How] 
 \textless{}uri\_prefix\textgreater{}../../pictures\textless{}/uri\_prefix\textgreater{}
\end{description}

\subsubsection{screen/screenelements-all.apx}
\begin{description}
\item[Short] 
 example of a multitude of screenelements
\item[Description] 

\item[How] 

\end{description}

\subsubsection{screen/screenelements-allbutflash.apx}
\begin{description}
\item[Short] 
 assembly of different kinds of screenelements
\item[Description] 
 Different labels and buttons are shown on the screen. First row containg a button and answerlabel, second row containing a label and a parameterlist, last row a textedit and picture
\item[How] 
 \textless{}button\textgreater{}, \textless{}answerlabel\textgreater{}, \textless{}label\textgreater{}, \textless{}parameterlist\textgreater{}, \textless{}textEdit\textgreater{}, \textless{}picture\textgreater{}
\end{description}

\subsubsection{screen/screenelements-answerlabel.apx}
\begin{description}
\item[Short] 
 shows the use of an answerlabel
\item[Description] 
 An answerlabel is shown on the screen, containing the correct answer for the current trial.
\item[How] 
 add \textless{}answerlabel\textgreater{} to \textless{}gridLayout\textgreater{}
\end{description}

\subsubsection{screen/screenelements-button-disabled.apx}
\begin{description}
\item[Short] 
 shows how to disable a button
\item[Description] 
 Two buttons are shown on the screen: nothing happens when clicking the disabled one
\item[How] 
 \textless{}disabled\textgreater{} true
\end{description}

\subsubsection{screen/screenelements-checkbox.apx}
\begin{description}
\item[Short] 
 shows how to use a checkbox
\item[Description] 
 Checkboxes are shown on the screen: if a checkbox is clicked, the subject is assumed to have responded to the trial
\item[How] 
 \textless{}checkBox\textgreater{}
\end{description}

\subsubsection{screen/screenelements-html.apx}
\begin{description}
\item[Short] 
 how to use a html page
\item[Description] 
 a html page is shown on the screen
\item[How] 
 \textless{}html\textgreater{} \textless{}page\textgreater{}
\end{description}

\subsubsection{screen/screenelements-label.apx}
\begin{description}
\item[Short] 
 shows how to add text to a label
\item[Description] 
 screen shows a large label filled with text
\item[How] 
 \textless{}label\textgreater{} \textless{}text\textgreater{}
\end{description}

\subsubsection{screen/screenelements-label-bgcolor.apx}
\begin{description}
\item[Short] 
 shows how to change background and font color of a label/button
\item[Description] 
 screen shows different labels/buttons with a different background color or font color
\item[How] 
 \textless{}bgcolor\textgreater{} changes the background color, \textless{}fgcolor\textgreater{} changes the font color
\end{description}

\subsubsection{screen/screenelements-label-html.apx}
\begin{description}
\item[Short] 
 shows how to use a bold font by using html code
\item[Description] 
 screen shows the text written in html code
\item[How] 
 \&lt;b\textgreater{}  text  \&lt;/b\textgreater{}
\end{description}

\subsubsection{screen/screenelements-label-resize.apx}
\begin{description}
\item[Short] 
 shows how to resize the text on a label
\item[Description] 
 screen shows different labels, with and without resizing of the text
\item[How] 
 \textless{}text\textgreater{} split text into different lines with 'enter' \& '\&\#xd;
\end{description}

\subsubsection{screen/screenelements-matrix.apx}
\begin{description}
\item[Short] 
 example of a matrix layout for buttons
\item[Description] 
 Screen shows buttons in a 2x2 matrix. After listening to the stimulus, you have to select one button in each column before clicking OK.
\item[How] 
 \textless{}matrix\textgreater{}
\end{description}

\subsubsection{screen/screenelements-parameterlabel.apx}
\begin{description}
\item[Short] 
 shows how to implement a parameterlabel on the screen
\item[Description] 
 A parameterlabel is shown on the screen, i.e. a label containg the value of a parameter.
\item[How] 
 screen: \textless{}parameterlabel\textgreater{}, stimuli: define parameter
\end{description}

\subsubsection{screen/screenelements-parameterlist.apx}
\begin{description}
\item[Short] 
 shows how to implement a parameterlist on the screen
\item[Description] 
 Parameters of the played stimuli are shown on the screen. This example shows two different parameterlists (two different trials).
\item[How] 
 \textless{}parameterlist\textgreater{}
\end{description}

\subsubsection{screen/screenelements-picture-changepicture.apx}
\begin{description}
\item[Short] 
 example of a picture-button
\item[Description] 
 Click the picture after listening to the stimulus
\item[How] 
 \textless{}picture\textgreater{}
\end{description}

\subsubsection{screen/screenelements-picture-notfound.apx}
\begin{description}
\item[Short] 
 shows error message for picture
\item[Description] 
 experiment doesn't load, error message: picture not found
\item[How] 
 wrong \textless{}uri\textgreater{} ('abc'-picture doesn't exist, can't be found in the pictures-folder)
\end{description}

\subsubsection{screen/screenelements-pictures.apx}
\begin{description}
\item[Short] 
 shows how to implement different pictures in the screen
\item[Description] 
 4 pictures are shown
\item[How] 
 \textless{}picture\textgreater{}
\end{description}

\subsubsection{screen/screenelements-pictures-disabled.apx}
\begin{description}
\item[Short] 
 shows how to disable a picture-button
\item[Description] 
 4 pictures are shown on the screen, one is disabled (number 4)
\item[How] 
 \textless{}picturelabel\textgreater{} instead of \textless{}picture\textgreater{} + picture 4 is not included in the button group
\end{description}

\subsubsection{screen/screenelements-pictures-feedback.apx}
\begin{description}
\item[Short] 

\item[Description] 

\item[How] 

\end{description}

\subsubsection{screen/screenelements-pictures-invalid.apx}
\begin{description}
\item[Short] 
 shows error message for picture
\item[Description] 
 experiment doesn't load, error message: picture invalid
\item[How] 
 wrong file extension for picture (xxx in stead of .png)
\end{description}

\subsubsection{screen/screenelements-pictures-resize2.apx}
\begin{description}
\item[Short] 
 example of screen layout with pictures
\item[Description] 
 Screen shows 4 pictures, after hearing a stimulus you should click one
\item[How] 
 \textless{}picture\textgreater{}
\end{description}

\subsubsection{screen/screenelements-pictures-resize.apx}
\begin{description}
\item[Short] 
 example of screen layout with pictures
\item[Description] 
 Screen shows 4 pictures, after hearing a stimulus you should click one
\item[How] 
 \textless{}picture\textgreater{}
\end{description}

\subsubsection{screen/screenelements-samename.apx}
\begin{description}
\item[Short] 
 Check whether everything is OK when two screenelements in different screens have the same name
\item[Description] 
 /
\item[How] 
 /
\end{description}

\subsubsection{screen/screenelements-slider.apx}
\begin{description}
\item[Short] 
 shows how to implement a vertical slider on the screen
\item[Description] 
 Slider is shown on the screen. Sliding the bar doens't change anything to the presented stimuli.
\item[How] 
 \textless{}slider\textgreater{}
\end{description}

\subsubsection{screen/screenelements-slider-horizontal.apx}
\begin{description}
\item[Short] 
 shows how to implement a horizontal slider on the screen
\item[Description] 
 Slider is shown on the screen. Sliding the bar doens't change anything to the presented stimuli.
\item[How] 
 \textless{}slider\textgreater{}
\end{description}

\subsubsection{screen/screenelements-spinbox.apx}
\begin{description}
\item[Short] 
 example of a spinbox
\item[Description] 
 a spinbox is an input field that only accepts numbers and has 2 buttons to respectively increase or decrease its value
\item[How] 
 add \textless{}spinBox\textgreater{} to Layout, define: startvalue, minimum \& maximum
\end{description}

\subsubsection{screen/screenelements-spinbox-parameter.apx}
\begin{description}
\item[Short] 
 example of a spinbox with adjustable parameter
\item[Description] 
 a spinbox is an input field that only accepts numbers and has 2 buttons to respectively increase or decrease its value
\item[How] 
 add \textless{}spinBox\textgreater{} to Layout, define: startvalue, minimum \& maximum, parameter to be adjusted
\end{description}

\subsubsection{screen/screenplugin-emptyresult.apx}
\begin{description}
\item[Short] 
 shows example of error message
\item[Description] 
 Demonstrate a screen generated by javascript, show error message because of invalid xml
\item[How] 
 wrong naming of source \textless{}pluginscreens\textgreater{}
\end{description}

\subsubsection{screen/stylesheet-elements.apx}
\begin{description}
\item[Short] 
 example of a different screenstyle
\item[Description] 
 screen has an other background color, buttons changed color, highlighting changed, ...
\item[How] 
 \textless{}style\_apex\textgreater{}
\end{description}

\subsubsection{screen/textinput-font.apx}
\begin{description}
\item[Short] 
 example of font input
\item[Description] 
 screen shows two 'input fields' with different fonts
\item[How] 
 \textless{}textEdit\textgreater{} \textless{}font\textgreater{}
\end{description}

\subsubsection{screen/textinput-inputmasks.apx}
\begin{description}
\item[Short] 
 example of various restricted inputs
\item[Description] 
 Screen shows four 'entry fields', but input is restricted (for example, only numbers allowed)
\item[How] 
 \textless{}textEdit\textgreater{}, \textless{}inputmask\textgreater{} (Defined in the Qt documentation!)
\end{description}

\subsubsection{screen/textinput-setfocus.apx}
\begin{description}
\item[Short] 
 example of restricted input
\item[Description] 
 Screen shows two 'entry fields', but input is restricted (for example, only numbers allowed)
\item[How] 
 \textless{}textEdit\textgreater{}, \textless{}inputmask\textgreater{} numbers = only numeric input will be allowed
\end{description}

\subsection{screenplugin}
\subsubsection{screenplugin/screenplugin-emptyresult.apx}
\begin{description}
\item[Short] 
\item[Description] 
\item[How] 
\end{description}

\subsection{soundcard}
\subsubsection{soundcard/soundcard-asio-multiplecard.apx}
\begin{description}
\item[Short] 
 Demonstrate the usage of multiple soundcards
\item[Description] 
 two soundcards are defined in \textless{}devices\textgreater{}
\item[How] 
 an extra soundcard is defined in \textless{}devices\textgreater{}, each with their own parameters
\end{description}

\subsubsection{soundcard/soundcard-cardname-inline.apx}
\begin{description}
\item[Short] 
 Specify the name of the sound card to be used
\item[Description] 
 Specify the sound card to be used.
\item[How] 
 The name of the sound card to be used can be specified in devices/device/card.  A list of devices in your system can be obtained by running apex from the commandline with parameter --soundcards
\end{description}

\subsubsection{soundcard/soundcard-clipping.apx}
\begin{description}
\item[Short] 
 Demonstrate what happens when the output clips
\item[Description] 
 The output clips of the first two trials of this experiment
\item[How] 
 In the results file, the following will be shown to indicate clipping occured: \textless{}device id="soundcard"\textgreater{}  \textless{}clipped channel="0"/\textgreater{} \textless{}/device\textgreater{}
\end{description}

\subsubsection{soundcard/soundcard-defaultsettings.apx}
\begin{description}
\item[Short] 
 Demonstrate default wavdevice settings
\item[Description] 
 Shows default settings in 'devices'
\item[How] 
 Devices: type of soundcard, number of channels (2), gain (0), samplerate (44100)
\end{description}

\subsubsection{soundcard/soundcard-driver-asio.apx}
\begin{description}
\item[Short] 
 Demonstrate the usage of different drivers and their parameters
\item[Description] 
 In this example asio is used as a driver (basicwav-coreaduio/jack/portaudio show examples of different drivers)
\item[How] 
 Devices: \textless{}driver\textgreater{}asio\textless{}/driver\textgreater{}
\end{description}

\subsubsection{soundcard/soundcard-driver-coreaudio.apx}
\begin{description}
\item[Short] 
 Demonstrate the usage of different drivers and their parameters
\item[Description] 
 In this example coreaudio is used as a driver (basicwav-asio/jack/portaudio show examples of different drivers)
\item[How] 
 Devices: \textless{}driver\textgreater{}coreaudio\textless{}/driver\textgreater{}
\end{description}

\subsubsection{soundcard/soundcard-driver-jack.apx}
\begin{description}
\item[Short] 
 Demonstrate the usage of different drivers and their parameters
\item[Description] 
 In this example jack is used as a driver (basicwav-coreaduio/asio/portaudio show examples of different drivers)
\item[How] 
 Devices: \textless{}driver\textgreater{}jack\textless{}/driver\textgreater{}
\end{description}

\subsubsection{soundcard/soundcard-driver-portaudio.apx}
\begin{description}
\item[Short] 
 Demonstrate the usage of different drivers and their parameters
\item[Description] 
 In this example portaudio is used as a driver (basicwav-coreaduio/jack/asio show examples of different drivers)
\item[How] 
 Devices: \textless{}driver\textgreater{}asio\textless{}/driver\textgreater{}
\end{description}

\subsubsection{soundcard/soundcard-gain-invalidchannel.apx}
\begin{description}
\item[Short] 
 Demonstrate the misapplication of adjusting gain to the wrong channel
\item[Description] 
 Gain is applied to the wrong channel (channel = 10, when there are only 2 channels)
\item[How] 
 Devices:  \textless{}gain channel="10"\textgreater{}0\textless{}/gain\textgreater{} -\textgreater{} it should be channel="0" of channel="1", because there are only 2 channels defined.
\end{description}

\subsubsection{soundcard/soundcard-mono-2channel.apx}
\begin{description}
\item[Short] 
 Demonstrate what happens when you use wavdevice with one channel not connected
\item[Description] 
 When you only use one of two defined channels you will not get an error, only a message: "not every inputchannel of soundcard is connected"
\item[How] 
 Devices: 2 channels defined, only 1 used in \textless{}connections\textgreater{}
\end{description}

\subsubsection{soundcard/soundcard-portaudio-6channels.apx}
\begin{description}
\item[Short] 
 Demonstrate the usage of multiple channels
\item[Description] 
 Add more channels to your experiment
\item[How] 
 set more channels (e.g. \textless{}channels\textgreater{}6\textless{}/channels, for 6 channels\textgreater{} and make connections with them so that the right stimulus is send to the right channel
\end{description}

\subsubsection{soundcard/soundcard-portaudio-bigbuffersize.apx}
\begin{description}
\item[Short] 

\item[Description] 

\item[How] 
 \textless{}buffersize\textgreater{}8192\textless{}/buffersize\textgreater{} is used
\end{description}

\subsubsection{soundcard/soundcard-portaudio-cardnotfound.apx}
\begin{description}
\item[Short] 
 Demonstrate the misapplication of \textless{}card\textgreater{}
\item[Description] 
 \textless{}card\textgreater{} specifies the name of the sound card. If you use the wrong cardname, you will get an error when you try to open your experiment in apex.
\item[How] 
 wrong card name used in \textless{}card\textgreater{} -\textgreater{} 'Soundmax' should be default or Soundmax should be defined in apexconfig.xml
\end{description}

\subsubsection{soundcard/soundcard-portaudio-interactivecard.apx}
\begin{description}
\item[Short] 
 Demonstrate the misapplication of \textless{}card\textgreater{}
\item[Description] 
 \textless{}card\textgreater{} specifies the name of the sound card to be used. If you use the wrong cardname, you will get an error when you try to open your experiment in apex.
\item[How] 
 wrong card name used in \textless{}card\textgreater{} -\textgreater{} 'invalid' should be default or an other name defined in apexconfig.xml
\end{description}

\subsubsection{soundcard/soundcard-wav-padzero.apx}
\begin{description}
\item[Short] 
 Demonstrate the use of padzero
\item[Description] 
 Add the given number of empty (filled with zero's) buffers to the output on the end of a stream. This avoids dropping of the last frames on some soundcards.
\item[How] 
 Devices: Add padzero function (e.g. \textless{}padzero\textgreater{}2\textless{}/padzero\textgreater{})
\end{description}

\subsection{xmlplugin}
\subsubsection{xmlplugin/autostimulus.apx}
\begin{description}
\item[Short] 
 Demonstrate datablocks and stimuli generated by javascript
\item[Description] 
 The datablocks and stimuli are automatically generated in this experiment using javascript
\item[How] 
 using plugindatablocks and pluginstimuli, setting the pluginparameters and referring to the wanted javascript in \textless{}general\textgreater{} (here autostimulus.js)
\end{description}

\subsubsection{xmlplugin/autotrials.apx}
\begin{description}
\item[Short] 
 Demonstrate trials, datablocks and stimuli generated by javascript
\item[Description] 
 The trials, datablocks and stimuli are automatically generated in this experiment using javascript
\item[How] 
 using plugintrials, plugindatablocks and pluginstimuli, setting the pluginparameters and referring to the wanted javascript in \textless{}general\textgreater{} (here autostimulus.js)
\end{description}

\subsubsection{xmlplugin/datablockplugin.apx}
\begin{description}
\item[Short] 
 Demonstrate a datablock generated by javascript
\item[Description] 
 The datablock is automatically generated in this experiment using javascript
\item[How] 
 using plugindatablocks and referring to the wanted javascript  in \textless{}datablocks\textgreater{} (here datablockgenerator.js)
\end{description}

\subsubsection{xmlplugin/datablockplugin\_l34.apx}
\begin{description}
\item[Short] 
 Demonstrate generating XML code for L34 directly from plugindatablocks and pluginstimuli
\item[Description] 
 The datablocks are automatically generated in this experiment using javascript
\item[How] 
 using plugindatablocks, pluginstimuli, setting the pluginparameters and referring to the wanted javascript in \textless{}datablocks\textgreater{} and \textless{}stimuli\textgreater{} (here datablockgenerator\_l34.js)
\end{description}

\subsubsection{xmlplugin/screenplugin.apx}
\begin{description}
\item[Short] 
 Demonstrate a screen generated by javascript
\item[Description] 
 The screen is automatically generated in this experiment using javascript
\item[How] 
 using pluginscreens, setting the pluginparameters and referring to the wanted javascript in \textless{}screens\textgreater{} (here screengenerator.js)
\end{description}

\subsubsection{xmlplugin/screenplugin-inlibrary.apx}
\begin{description}
\item[Short] 
 Demonstrate the misapplication of a screen generated by javascript
\item[Description] 
 The screen is not generated in this experiment because the script source is invalid. You will get an error message when you try to open the experiment.
\item[How] 
 script source refers to \textless{}general\textgreater{} instead of the script itself. It should be: \textless{}script source="screengenerator-library"/\textgreater{} in \textless{}screens\textgreater{} OR \textless{}scriptlibrary\textgreater{}screengenerator-library.js\textless{}/scriptlibrary\textgreater{} in \textless{}general\textgreater{}
\end{description}

\subsubsection{xmlplugin/screenplugin-invalidscript.apx}
\begin{description}
\item[Short] 
 Demonstrate the misapplication of a screen generated by  javascript
\item[Description] 
 The screen is not generated in this experiment because the defined javascript is empty.
\item[How] 
 invalidxml.js is an empty/invalid javascript so no screen can be generated.
\end{description}

\subsubsection{xmlplugin/screenplugin-invalidxml.apx}
\begin{description}
\item[Short] 
 Demonstrate the misapplication of a screen generated by javascript
\item[Description] 
 The screen is not generated in this experiment because the defined javascript is empty. You will get an error when you try to open the experiment.
\item[How] 
 the javascript contains an error so no screen can be generated.
\end{description}

\subsubsection{xmlplugin/screenplugin-multiscreen.apx}
\begin{description}
\item[Short] 
 Demonstrate a screen generated by javascript
\item[Description] 
 The screen is automatically generated in this experiment using javascript
\item[How] 
 using pluginscreens, setting the pluginparameters and referring to the wanted javascript in \textless{}screens\textgreater{} (here screengenerator-multiscreen.js)
\end{description}

\subsubsection{xmlplugin/stimulusplugin.apx}
\begin{description}
\item[Short] 
 Demonstrate stimuli generated by javascript
\item[Description] 
 The stimuli are automatically generated in this experiment using javascript
\item[How] 
 using pluginstimuli, setting the pluginparameters and referring to the wanted javascript in \textless{}stimuli\textgreater{} (here stimulusgenerator.js)
\end{description}

